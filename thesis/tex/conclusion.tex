\chapter{Conclusion}
\label{ch:conclusion}

In this thesis, we conducted a detailed analysis of the original \nlmapstwo{}
dataset finding several shortcomings, many of which were introduced by a flawed
approach of generating synthetic data. We fixed these shortcomings as well as
possible and extended the dataset by generating a linguistically diverse dataset
using probabilistic templates. Training on the extended dataset greatly improves
accuracy on unseen data, especially by making the resulting parser robust
against new location names.

We built a web interface for issuing NL queries, which can also be used to
correct wrong parses and which is capable of training the parser on the new
feedback in an online fashion. With the help of hired annotators, we created the
first large \nlmaps{} dataset consisting of real user queries. This new dataset
was used to demonstrate the effectiveness of our online learning setup in
various simulations, although traditional offline learning still proved
superior.

In our experiments and discussion, we gained new insight into what the main
challenges of the \nlmaps{} task are and proposed various directions for
subsequent research.

%%% Local Variables:
%%% coding: utf-8
%%% mode: latex
%%% TeX-engine: xetex
%%% TeX-parse-self: t
%%% TeX-command-extra-options: "-shell-escape"
%%% TeX-master: "../thesis"
%%% End: