\chapter{Abriss (German Abstract)}
\label{ch:abstract-german}

OpenStreetMap (OSM) speichert riesige Mengen an Daten, die nützlich sind, um
alltägliche und weniger alltägliche Fragen zu beantworten. Doch für exakte
Abfragen sind in der Regel entweder auf Einzelfragen spezialisierte Software
oder Kenntnisse in speziellen OSM-Anfragesprachen notwendig. Allerdings kann mit
dem bestehenden Korpus \nlmapstwo{} ein Parsing-System trainiert werden, das
natürlichsprachliche Anfragen in eine maschinenlesbare Anfrage übersetzt,
mithilfe derer dann die Antwort auf die Frage in der OSM-Datenbank gefunden
werden kann. Die Zuverlässigkeit von auf diesem Korpus trainierten
Parsing-Systemen ist aber sehr begrenzt.

In dieser Arbeit analysieren wir das Korpus \nlmapstwo{} und stellen
verschiedene Mängel fest. Nach dem Beheben dieser Mängel erweitern wir das
Korpus um neue, linguistisch diverse Anfragen, die mit einem probabilistischen
Mustervorlagensystem erstellt werden. Das erweiterte Korpus nutzen wir dann, um
ein neues Parsing-System zu trainieren, das die bisher in der Literatur
vorgestellten Systeme deutlich übertrifft.

Unser Parsing-System machen wir in einem neuen Web-Interface zugänglich, das
sowohl dazu verwendet werden kann, Fragen zu stellen, als auch dazu, falsche
Antworten zu korrigieren. Das Web-Interface ist außerdem dazu in der Lage, das
Parsing-System auf korrigierten Beispielen online weiter zu trainieren. Wir
nutzen es in dieser Arbeit auch, um mithilfe von Studienteilnehmern das erste
große \nlmaps{}-Korpus zu erstellen, das aus Anfragen verschiedener Menschen
besteht. Dieses Korpus wird in dieser Arbeit zudem dazu verwendet, in
verschiedenen Online-Lern-Simulationen den besten Ansatz zu finden, von neuen
Rückmeldungen zu lernen.

%%% Local Variables:
%%% coding: utf-8
%%% mode: latex
%%% TeX-engine: xetex
%%% TeX-parse-self: t
%%% TeX-command-extra-options: "-shell-escape"
%%% TeX-master: "../thesis"
%%% End: