\chapter{Annotation Guidelines}
\label{ch:annotation-guidelines}

\section{Requirements}

The annotation website\footnote{\url{https://nlmaps.gorgor.de/}} is meant to be
used with a recent version of a modern Browser. That means Firefox, Safari,
Chrome or any other Chromium derivates (such as recent MS Edge). It is not
optimized for mobile use, so please use a desktop computer.

\section{Principles}

In essence, we need a dataset that fulfills three criteria:

\begin{enumerate}
\item The natural language (NL) queries should be linguistically diverse, i.e. a mix of short search-engine-style queries and full questions like you would ask another person.
\item The queries should cover a large part of the commonly used OpenStreetmap (OSM) tags that are relevant for our system.
\item The queries should cover the most useful OSM tags in particular depth. E.g., asking for restaurants and shops is arguably one of the most useful areas.
\end{enumerate}

\section{Linguistic Diversity}

When entering queries, please diversify your language use. Valid variations of
the same question include:

\begin{itemize}
\item \nl{closest swimming pool near Eiffel Tower in Paris}
\item \nl{In Paris, give me the swimming pool that is closest to the Eiffel Tower!}
\item \nl{Closest place where I can take a swim near Eiffel Tower in Paris}
\end{itemize}

However, keep it natural and don’t make your queries artificially complex. If a
particular query style comes more natural to you, it’s alright to use it more
often. Just make sure that not all of you queries look alike.

Similarly, avoid using the same place name more than a couple of times. Also use
place names in other languages than German and English, e.g. \nl{České
  Budějovice} or \nl{València}.

It’s \textbf{very important} that you don’t only base your wording on the name
of the OSM tags. E.g., for \osmtag{highway=speed_camera} you can ask for a
\nl{speed camera}, but you can also ask for a \nl{speed trap} or a \nl{radar
  trap}.

\section{Tag Diversity and Depth}

Take a \nl{quick} look at the most important OSM
features\footnote{\url{https://wiki.openstreetmap.org/wiki/Map_features}} to get
a feel for what things you can ask for in OSM. You can use this as an
inspiration if you run out of ideas about what to ask. Choose tags that you find
relevant.

\textbf{Beware of the \osmtag{building=*} tag!} It is used to tag what the
building was built as, not to tag its current use. E.g., a place tagged
\osmtag{building=church} may not be a church anymore; churches are tagged with
the tags \osmtag{amenity=place_of_worship} + \osmtag{religion=christian}. If
in doubt, don’t use the \osmtag{building=*} tag, at all.

In general, enter more than one query for a chosen tag or tag combination,
especially if the system fails answering the query correctly.

\subsection{Most Relevant Keys}

Please enter at least the given amount of queries for each of the following. The
linked wiki pages give you a feel for what the most common tags in each category
are.

\begin{itemize}
\item \osmtag{shop=*}: 30.
\item \osmtag{leisure=*}: 20.
\item \osmtag{sport=*}: 20.
\item \osmtag{craft=*}: 20.
\item \osmtag{man_made=*}: 10.
\item \osmtag{amenity=cafe}: 10.
\item \osmtag{amenity=restaurant}: 10.
\item \osmtag{amenity=fast_food}: 10.
\item \osmtag{cuisine=*}: 20.
\item \osmtag{diet:*=*}: 20.
\item \osmtag{wheelchair=yes}: 20. Just sprinkle phrases like
  \nl{wheelchair-accessible [place]} or \nl{[places] that are
    wheelchair-accessible} into your queries now and then.
\end{itemize}

Some tags \emph{can and should} be combined. E.g., use
\mrl{shop=massage,wheelchair=yes} for wheelchair-accessible massage shops or
\mrl{club=sport,sport=tennis} for tennis clubs. But use only
\osmtag{sport=tennis} if you’re just asking for places to play tennis at.

Especially the \osmtag{cuisine=*} and \osmtag{diet:*=*} tags can be combined
productively. Some examples:

\begin{itemize}
\item \osmtag{cuisine=japanese}: Places serving japanese food
\item \mrl{cuisine=japanese,amenity=fast_food}: Fast food restaurants
  serving\\japanese food
\item \mrl{or(diet:vegan=yes,diet:vegan=only),amenity=cafe}: Vegan cafes
\item \mrl{or(diet:gluten_free=yes,diet:gluten_free=only),}\\
  \mrl{cuisine=burger,amenity=restaurant}: Restaurants serving\\gluten-free
  burgers
\end{itemize}

\section{Miscellaneous}

\begin{itemize}
\item Sometimes deciding between QType \mrl{findkey('name')} and \mrl{latlong} is not obvious. By convention:
  \begin{itemize}
  \item \nl{\emph{Which/What} restaurants/museums/etc. …}: \mrl{findkey('name')}
  \item \nl{\emph{Name} restaurants/museums/etc. …}: \mrl{findkey('name')}
  \item \nl{What are restaurants/museums/etc. … \emph{called}?}: \mrl{findkey('name')}
  \item \nl{\emph{Give/Tell} (me) the \emph{names} of restaurants/museums/etc.
      …}:\\\mrl{findkey('name')}
  \item \nl{\emph{Show/Give/Tell} (me) restaurants/museums/etc. …}: \mrl{latlong}
  \item \nl{\emph{Where} are restaurants/museums/etc. …}: \mrl{latlong}
  \item \nl{\emph{Location/Coordinates} of restaurants/museums/etc. …}: \mrl{latlong}
  \end{itemize}
\item Don’t repeat the same query with only a different location. Adjust the wording, as well.
\item Some queries will not return results even if they are correct (e.g. rare
  tags like gluten-free etc.). Please base your judgement primarily on the mrl,
  not on the answer or map.
\item Avoid querying too much data (\nl{trees in Berlin}) or too large areas
  (\nl{restaurants in Bangladesh}) to put less stress on servers and your
  browser.
\item \nl{Show all restaurants in X that are wheelchair-accessible!}: Target
  tags include \mrl{wheelchair=yes}, QType is `latlong`
\item \nl{Is X accessible by wheelchair?}: Use QType
  \mrl{findkey('wheelchair')}, no \mrl{wheelchair=*} target tag
\item Questions looking for the closest thing to some other thing should always
  have a maxdist of \mrl{DIST_INTOWN}. In theory, this doesn’t make sense. It’s
  just a limitation of the current system.
\item Use the appropriate maxdist value according to the table in chapter 4 of
  the tutorial\footnote{\url{https://nlmaps.gorgor.de/tutorial?chapter=4}}.
  E.g., when using the word “near” in your query, use \mrl{DIST_INTOWN}.
\end{itemize}

\section{Counting Annotations}

A natural language query and the corresponding mrl are considered one
\emph{complete annotation}. If you don’t know what the correct mrl looks like,
you can choose \enquote{Wrong, but I cannot help} and it will be considered an
\emph{incomplete annotation}. This is still valuable; four incomplete
annotations will count as one complete annotation.

%%% Local Variables:
%%% coding: utf-8
%%% mode: latex
%%% TeX-engine: xetex
%%% TeX-parse-self: t
%%% TeX-command-extra-options: "-shell-escape"
%%% TeX-master: "../thesis"
%%% End: