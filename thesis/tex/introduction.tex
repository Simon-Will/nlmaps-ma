\chapter{Introduction}
\label{sec:introduction}

The OpenStreetMap projects’ database stores a wealth of geographical information
about the world including detailed information about mapped places. However,
extracting specific information – such as the answer for the simple natural
language question \nl{Which Italian restaurants in Berlin are
  wheelchair-accessible?} – requires either purpose-built tooling or knowledge
of custom OpenStreetMap query languages. A natural language interface to the
database would make the available information more accessible.

Important groundwork for such a natural language interface was laid by Lawrence
and Riezler, who developed a simple OpenStreetMap machine-readable query
language (MRL) and also released the NLMaps dataset mapping natural language
queries to their corresponding counterpart in that query language.

This thesis reviews previous work on NLMaps and reveals shortcomings in the
published NLMaps dataset. In order to improve on it, the existing NLMaps dataset
is overhauled by eliminating some of the identified shortcomings and by
extending it through the use of probabilistic templates to include more diverse
queries on both the natural language and the machine-readable language side.

The improved NLMaps dataset is used to train an improved parsing model exposed
via a new web interface that can be used for both asking queries and correcting
queries. In order to further improve the model, the system is able to directly
learn from the corrected queries using an online learning technique.

An annotation experiment is performed, in which people use the new web interface
for asking and correcting queries. The data collected in this way is used to
test the online learning setup and is also released as a new NLMaps dataset.

%%% Local Variables:
%%% coding: utf-8
%%% mode: latex
%%% TeX-engine: xetex
%%% TeX-parse-self: t
%%% TeX-command-extra-options: "-shell-escape"
%%% TeX-master: "../thesis"
%%% End: