\chapter{Abstract}
\label{ch:abstract-english}

OpenStreetMap (OSM) stores a large amount of data useful for everyday tasks as
well as for informational queries, but it is difficult to query without using
specialized applications or being versed in special OSM query languages. The
existing \nlmapstwo{} dataset can be used for building a question answering
system that translates a natural language query into a machine-readable query
used for extracting the answer from the OSM database, but the performance of
parsers trained on that dataset has been very limited when tested on new
queries.

In this thesis, we analyze \nlmapstwo{} and find several shortcomings. After
fixing them, we extend the dataset by generating new, linguistically diverse
queries with a probabilistic templating approach, which we then use to train a
new parser that significantly outperforms parsers presented in previous work.

In order to make our parser accessible, we build a web interface for asking
queries, which can also be used to correct wrong parses and which is capable of
learning from the corrections in an online fashion. We use the new interface to
hire users to issue queries and to correct the parses, thus creating the first
large \nlmaps{} dataset consisting of real user queries. This dataset is used in
online learning simulation experiments in order to find the most effective
approach to learn from new feedback.

%%% Local Variables:
%%% coding: utf-8
%%% mode: latex
%%% TeX-engine: luatex
%%% TeX-parse-self: t
%%% TeX-command-extra-options: "-shell-escape"
%%% TeX-master: "../thesis"
%%% End:
