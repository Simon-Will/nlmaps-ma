\chapter{Experiments}
\label{ch:experiments}

Our goal is simulating different strategies of online learning for the new web
interface. Before we can do that, we first pre-train parsers in
Section~\ref{sec:pre-training} on the existing datasets \nlmapstwo{} and the
variations we introduced in \nlmapsthree{} and evaluate whether the dataset
extensions yield any improvement in parser quality. In
Section~\ref{sec:annotation}, we then hire annotators to use our web interface
to ask NL queries and correct the MRL parse if it is incorrect, thus collecting
a new dataset consisting of real user queries. Finally, in
Section~\ref{sec:online-simulation} we use the newly acquired dataset for
evaluating various online learning setups.

For all our experiments, we use the same model architecture: A character-based
one-layer bidirectional GRU encoder-decoder \parencite{cho-2014} model with
attention.\parencite{bahdanau-2015} The dimension of both source and target
embeddings is 620, the encoder layer size is 500, the decoder layer size is 1000
and we don’t use dropout. This model configuration is adopted from
\textcite{staniek-2020}.

\section{Training on \nlmapstwo{} and \nlmapsthree{}}
\label{sec:pre-training}

While \textcite{staniek-2020} trained his model on the \nlmapstwo{} dataset for
100 epochs (of 16172 instances each), we train our models in this section for a
shorter time, which is still enough for sufficient convergence: The model on
\nlmapstwoone{} is trained for 60 epochs (of 15113 instances each), while the
models on variations of \nlmapsthree{} are trained for 30 epochs (of 30226
instances each). All models are trained with the Adam optimizer (\(\beta_1 =
0.9\) and \(\beta_2 = 0.999\)) and a learning rate of \num{0.0002}

Table~\ref{tab:pre-trained-performance} shows the results of testing the models
on the different variations of \nlmaps{} datasets. Recall: \nlmtwoone{} is the
result of fixing issues in the MRL queries of \nlmtwo{}, but has identical NL
queries; \nlmthreea{} is purely generated with probabilistic templates and
\nlmthreeb{} is its counterpart with added noise; \nlmthreenormal{} is
\nlmtwoone{} + \nlmthreea{} and \nlmthree{} is \nlmtwoone{} + \nlmthreeb{}. The
datasets \nlmfourraw{} and \nlmfour{} are introduced in
Section~\ref{sec:annotation} and are the user-supplied MRL-NL pairs with some
corrections applied to them in \nlmfour{}. Even though the last two datasets
were not yet available when the models were pre-trained, we still evaluate the
pre-trained models on them in this section because we consider the results on
these user-supplied queries the most relevant.

\begin{table}[h]
  \centering
  \begin{tabular}{lcccccccc}
    \toprule
    \diagbox{Train}{Test} & \nlmtwo{} & \nlmtwoone{} & \nlmthreea{} & \nlmthreeb{} & \nlmthreenormal{} & \nlmthree{} & \nlmfourraw{} & \nlmfour{}\\
    \midrule
    \textcite{staniek-2020} & \bfnum{0.898} & \num{0.844} & \num{0.039} & \num{0.033} & \num{0.441} & \num{0.439} & \num{0.050} & \num{0.052}\\
    \nlmtwoone{} & \num{0.783} & \num{0.913} & \num{0.034} & \num{0.029} & \num{0.474} & \num{0.471} & \num{0.070} & \num{0.069}\\
    \nlmthreea{} & \num{0.224} & \num{0.251} & \bfnum{0.987} & \num{0.789} & \num{0.618} & \num{0.519} & \num{0.217} & \num{0.223}\\
    \nlmthreeb{} & \num{0.372} & \num{0.424} & \num{0.976} & \bfnum{0.884} & \num{0.700} & \num{0.656} & \num{0.226} & \num{0.233}\\
    \nlmthreenormal{} & \num{0.790} & \bfnum{0.919} & \num{0.978} & \num{0.792} & \bfnum{0.948} & \num{0.857} & \bfnum{0.307} & \bfnum{0.311}\\
    \nlmthree{} & \num{0.787} & \num{0.913} & \num{0.950} & \num{0.834} & \num{0.931} & \bfnum{0.874} & \num{0.281} & \num{0.289}\\
    \bottomrule
  \end{tabular}
  \caption{Performance of pre-trained parsers}
  \label{tab:pre-trained-performance}
\end{table}

Unsurprisingly, the model by \textcite{staniek-2020}, which was trained on
\nlmapstwo{}, performs best on the corresponding test set with an accuracy of
\SI{89.8}{\%}. The model trained on \nlmtwoone{}, which contains the fixes of
tag usage and inconsistencies described in Chapter~\ref{ch:nlmaps-improvement},
achieves a higher accuracy on its corresponding test set with \SI{91.3}{\%}.
This is to be expected since the resolution of inconsistencies in MRL structure
and tag usage makes a new part of the dataset accessible for confident
predictions in the first place. These two models’ performance drops dramatically
on new datasets. An error analysis (see Figure~\ref{fig:v21-errors}) shows that
not only does the model trained on \nlmtwoone{} make an error in the
\mrl{target_nwr} operator in over \SI{70}{\%} of queries from the \nlmapsfour{}
test set, the recognized \mrl{area} is false in more than half the queries, as
well.

\begin{figure}[h]
  \centering
  \resizebox{\textwidth}{!}{%% Creator: Matplotlib, PGF backend
%%
%% To include the figure in your LaTeX document, write
%%   \input{<filename>.pgf}
%%
%% Make sure the required packages are loaded in your preamble
%%   \usepackage{pgf}
%%
%% and, on pdftex
%%   \usepackage[utf8]{inputenc}\DeclareUnicodeCharacter{2212}{-}
%%
%% or, on luatex and xetex
%%   \usepackage{unicode-math}
%%
%% Figures using additional raster images can only be included by \input if
%% they are in the same directory as the main LaTeX file. For loading figures
%% from other directories you can use the `import` package
%%   \usepackage{import}
%%
%% and then include the figures with
%%   \import{<path to file>}{<filename>.pgf}
%%
%% Matplotlib used the following preamble
%%   \usepackage{fontspec}
%%   \setmainfont{LinLibertine_R.otf}[Path=/usr/share/texmf-dist/fonts/opentype/public/libertine/]
%%   \setsansfont{LinLibertine_R.otf}[Path=/usr/share/texmf-dist/fonts/opentype/public/libertine/]
%%   \setmonofont{DejaVuSansMono.ttf}[Path=/home/gorgor/.virtualenvs/ma/lib/python3.9/site-packages/matplotlib/mpl-data/fonts/ttf/]
%%
\begingroup%
\makeatletter%
\begin{pgfpicture}%
\pgfpathrectangle{\pgfpointorigin}{\pgfqpoint{6.400000in}{3.500000in}}%
\pgfusepath{use as bounding box, clip}%
\begin{pgfscope}%
\pgfsetbuttcap%
\pgfsetmiterjoin%
\definecolor{currentfill}{rgb}{1.000000,1.000000,1.000000}%
\pgfsetfillcolor{currentfill}%
\pgfsetlinewidth{0.000000pt}%
\definecolor{currentstroke}{rgb}{1.000000,1.000000,1.000000}%
\pgfsetstrokecolor{currentstroke}%
\pgfsetdash{}{0pt}%
\pgfpathmoveto{\pgfqpoint{0.000000in}{0.000000in}}%
\pgfpathlineto{\pgfqpoint{6.400000in}{0.000000in}}%
\pgfpathlineto{\pgfqpoint{6.400000in}{3.500000in}}%
\pgfpathlineto{\pgfqpoint{0.000000in}{3.500000in}}%
\pgfpathclose%
\pgfusepath{fill}%
\end{pgfscope}%
\begin{pgfscope}%
\pgfsetbuttcap%
\pgfsetmiterjoin%
\definecolor{currentfill}{rgb}{1.000000,1.000000,1.000000}%
\pgfsetfillcolor{currentfill}%
\pgfsetlinewidth{0.000000pt}%
\definecolor{currentstroke}{rgb}{0.000000,0.000000,0.000000}%
\pgfsetstrokecolor{currentstroke}%
\pgfsetstrokeopacity{0.000000}%
\pgfsetdash{}{0pt}%
\pgfpathmoveto{\pgfqpoint{1.024000in}{0.490000in}}%
\pgfpathlineto{\pgfqpoint{6.272000in}{0.490000in}}%
\pgfpathlineto{\pgfqpoint{6.272000in}{3.430000in}}%
\pgfpathlineto{\pgfqpoint{1.024000in}{3.430000in}}%
\pgfpathclose%
\pgfusepath{fill}%
\end{pgfscope}%
\begin{pgfscope}%
\pgfpathrectangle{\pgfqpoint{1.024000in}{0.490000in}}{\pgfqpoint{5.248000in}{2.940000in}}%
\pgfusepath{clip}%
\pgfsetbuttcap%
\pgfsetmiterjoin%
\definecolor{currentfill}{rgb}{1.000000,0.000000,0.000000}%
\pgfsetfillcolor{currentfill}%
\pgfsetlinewidth{0.000000pt}%
\definecolor{currentstroke}{rgb}{0.000000,0.000000,0.000000}%
\pgfsetstrokecolor{currentstroke}%
\pgfsetstrokeopacity{0.000000}%
\pgfsetdash{}{0pt}%
\pgfpathmoveto{\pgfqpoint{1.024000in}{0.623636in}}%
\pgfpathlineto{\pgfqpoint{1.997139in}{0.623636in}}%
\pgfpathlineto{\pgfqpoint{1.997139in}{0.829231in}}%
\pgfpathlineto{\pgfqpoint{1.024000in}{0.829231in}}%
\pgfpathclose%
\pgfusepath{fill}%
\end{pgfscope}%
\begin{pgfscope}%
\pgfpathrectangle{\pgfqpoint{1.024000in}{0.490000in}}{\pgfqpoint{5.248000in}{2.940000in}}%
\pgfusepath{clip}%
\pgfsetbuttcap%
\pgfsetmiterjoin%
\definecolor{currentfill}{rgb}{1.000000,0.000000,0.000000}%
\pgfsetfillcolor{currentfill}%
\pgfsetlinewidth{0.000000pt}%
\definecolor{currentstroke}{rgb}{0.000000,0.000000,0.000000}%
\pgfsetstrokecolor{currentstroke}%
\pgfsetstrokeopacity{0.000000}%
\pgfsetdash{}{0pt}%
\pgfpathmoveto{\pgfqpoint{1.024000in}{1.034825in}}%
\pgfpathlineto{\pgfqpoint{1.322893in}{1.034825in}}%
\pgfpathlineto{\pgfqpoint{1.322893in}{1.240420in}}%
\pgfpathlineto{\pgfqpoint{1.024000in}{1.240420in}}%
\pgfpathclose%
\pgfusepath{fill}%
\end{pgfscope}%
\begin{pgfscope}%
\pgfpathrectangle{\pgfqpoint{1.024000in}{0.490000in}}{\pgfqpoint{5.248000in}{2.940000in}}%
\pgfusepath{clip}%
\pgfsetbuttcap%
\pgfsetmiterjoin%
\definecolor{currentfill}{rgb}{1.000000,0.000000,0.000000}%
\pgfsetfillcolor{currentfill}%
\pgfsetlinewidth{0.000000pt}%
\definecolor{currentstroke}{rgb}{0.000000,0.000000,0.000000}%
\pgfsetstrokecolor{currentstroke}%
\pgfsetstrokeopacity{0.000000}%
\pgfsetdash{}{0pt}%
\pgfpathmoveto{\pgfqpoint{1.024000in}{1.446014in}}%
\pgfpathlineto{\pgfqpoint{1.128265in}{1.446014in}}%
\pgfpathlineto{\pgfqpoint{1.128265in}{1.651608in}}%
\pgfpathlineto{\pgfqpoint{1.024000in}{1.651608in}}%
\pgfpathclose%
\pgfusepath{fill}%
\end{pgfscope}%
\begin{pgfscope}%
\pgfpathrectangle{\pgfqpoint{1.024000in}{0.490000in}}{\pgfqpoint{5.248000in}{2.940000in}}%
\pgfusepath{clip}%
\pgfsetbuttcap%
\pgfsetmiterjoin%
\definecolor{currentfill}{rgb}{1.000000,0.000000,0.000000}%
\pgfsetfillcolor{currentfill}%
\pgfsetlinewidth{0.000000pt}%
\definecolor{currentstroke}{rgb}{0.000000,0.000000,0.000000}%
\pgfsetstrokecolor{currentstroke}%
\pgfsetstrokeopacity{0.000000}%
\pgfsetdash{}{0pt}%
\pgfpathmoveto{\pgfqpoint{1.024000in}{1.857203in}}%
\pgfpathlineto{\pgfqpoint{4.833144in}{1.857203in}}%
\pgfpathlineto{\pgfqpoint{4.833144in}{2.062797in}}%
\pgfpathlineto{\pgfqpoint{1.024000in}{2.062797in}}%
\pgfpathclose%
\pgfusepath{fill}%
\end{pgfscope}%
\begin{pgfscope}%
\pgfpathrectangle{\pgfqpoint{1.024000in}{0.490000in}}{\pgfqpoint{5.248000in}{2.940000in}}%
\pgfusepath{clip}%
\pgfsetbuttcap%
\pgfsetmiterjoin%
\definecolor{currentfill}{rgb}{1.000000,0.000000,0.000000}%
\pgfsetfillcolor{currentfill}%
\pgfsetlinewidth{0.000000pt}%
\definecolor{currentstroke}{rgb}{0.000000,0.000000,0.000000}%
\pgfsetstrokecolor{currentstroke}%
\pgfsetstrokeopacity{0.000000}%
\pgfsetdash{}{0pt}%
\pgfpathmoveto{\pgfqpoint{1.024000in}{2.268392in}}%
\pgfpathlineto{\pgfqpoint{1.510570in}{2.268392in}}%
\pgfpathlineto{\pgfqpoint{1.510570in}{2.473986in}}%
\pgfpathlineto{\pgfqpoint{1.024000in}{2.473986in}}%
\pgfpathclose%
\pgfusepath{fill}%
\end{pgfscope}%
\begin{pgfscope}%
\pgfpathrectangle{\pgfqpoint{1.024000in}{0.490000in}}{\pgfqpoint{5.248000in}{2.940000in}}%
\pgfusepath{clip}%
\pgfsetbuttcap%
\pgfsetmiterjoin%
\definecolor{currentfill}{rgb}{1.000000,0.000000,0.000000}%
\pgfsetfillcolor{currentfill}%
\pgfsetlinewidth{0.000000pt}%
\definecolor{currentstroke}{rgb}{0.000000,0.000000,0.000000}%
\pgfsetstrokecolor{currentstroke}%
\pgfsetstrokeopacity{0.000000}%
\pgfsetdash{}{0pt}%
\pgfpathmoveto{\pgfqpoint{1.024000in}{2.679580in}}%
\pgfpathlineto{\pgfqpoint{3.790495in}{2.679580in}}%
\pgfpathlineto{\pgfqpoint{3.790495in}{2.885175in}}%
\pgfpathlineto{\pgfqpoint{1.024000in}{2.885175in}}%
\pgfpathclose%
\pgfusepath{fill}%
\end{pgfscope}%
\begin{pgfscope}%
\pgfpathrectangle{\pgfqpoint{1.024000in}{0.490000in}}{\pgfqpoint{5.248000in}{2.940000in}}%
\pgfusepath{clip}%
\pgfsetbuttcap%
\pgfsetmiterjoin%
\definecolor{currentfill}{rgb}{1.000000,0.000000,0.000000}%
\pgfsetfillcolor{currentfill}%
\pgfsetlinewidth{0.000000pt}%
\definecolor{currentstroke}{rgb}{0.000000,0.000000,0.000000}%
\pgfsetstrokecolor{currentstroke}%
\pgfsetstrokeopacity{0.000000}%
\pgfsetdash{}{0pt}%
\pgfpathmoveto{\pgfqpoint{1.024000in}{3.090769in}}%
\pgfpathlineto{\pgfqpoint{1.872021in}{3.090769in}}%
\pgfpathlineto{\pgfqpoint{1.872021in}{3.296364in}}%
\pgfpathlineto{\pgfqpoint{1.024000in}{3.296364in}}%
\pgfpathclose%
\pgfusepath{fill}%
\end{pgfscope}%
\begin{pgfscope}%
\pgfpathrectangle{\pgfqpoint{1.024000in}{0.490000in}}{\pgfqpoint{5.248000in}{2.940000in}}%
\pgfusepath{clip}%
\pgfsetbuttcap%
\pgfsetmiterjoin%
\definecolor{currentfill}{rgb}{0.756863,0.988235,0.756863}%
\pgfsetfillcolor{currentfill}%
\pgfsetlinewidth{0.000000pt}%
\definecolor{currentstroke}{rgb}{0.000000,0.000000,0.000000}%
\pgfsetstrokecolor{currentstroke}%
\pgfsetstrokeopacity{0.000000}%
\pgfsetdash{}{0pt}%
\pgfpathmoveto{\pgfqpoint{1.997139in}{0.623636in}}%
\pgfpathlineto{\pgfqpoint{6.272000in}{0.623636in}}%
\pgfpathlineto{\pgfqpoint{6.272000in}{0.829231in}}%
\pgfpathlineto{\pgfqpoint{1.997139in}{0.829231in}}%
\pgfpathclose%
\pgfusepath{fill}%
\end{pgfscope}%
\begin{pgfscope}%
\pgfpathrectangle{\pgfqpoint{1.024000in}{0.490000in}}{\pgfqpoint{5.248000in}{2.940000in}}%
\pgfusepath{clip}%
\pgfsetbuttcap%
\pgfsetmiterjoin%
\definecolor{currentfill}{rgb}{0.756863,0.988235,0.756863}%
\pgfsetfillcolor{currentfill}%
\pgfsetlinewidth{0.000000pt}%
\definecolor{currentstroke}{rgb}{0.000000,0.000000,0.000000}%
\pgfsetstrokecolor{currentstroke}%
\pgfsetstrokeopacity{0.000000}%
\pgfsetdash{}{0pt}%
\pgfpathmoveto{\pgfqpoint{1.322893in}{1.034825in}}%
\pgfpathlineto{\pgfqpoint{6.272000in}{1.034825in}}%
\pgfpathlineto{\pgfqpoint{6.272000in}{1.240420in}}%
\pgfpathlineto{\pgfqpoint{1.322893in}{1.240420in}}%
\pgfpathclose%
\pgfusepath{fill}%
\end{pgfscope}%
\begin{pgfscope}%
\pgfpathrectangle{\pgfqpoint{1.024000in}{0.490000in}}{\pgfqpoint{5.248000in}{2.940000in}}%
\pgfusepath{clip}%
\pgfsetbuttcap%
\pgfsetmiterjoin%
\definecolor{currentfill}{rgb}{0.756863,0.988235,0.756863}%
\pgfsetfillcolor{currentfill}%
\pgfsetlinewidth{0.000000pt}%
\definecolor{currentstroke}{rgb}{0.000000,0.000000,0.000000}%
\pgfsetstrokecolor{currentstroke}%
\pgfsetstrokeopacity{0.000000}%
\pgfsetdash{}{0pt}%
\pgfpathmoveto{\pgfqpoint{1.128265in}{1.446014in}}%
\pgfpathlineto{\pgfqpoint{6.272000in}{1.446014in}}%
\pgfpathlineto{\pgfqpoint{6.272000in}{1.651608in}}%
\pgfpathlineto{\pgfqpoint{1.128265in}{1.651608in}}%
\pgfpathclose%
\pgfusepath{fill}%
\end{pgfscope}%
\begin{pgfscope}%
\pgfpathrectangle{\pgfqpoint{1.024000in}{0.490000in}}{\pgfqpoint{5.248000in}{2.940000in}}%
\pgfusepath{clip}%
\pgfsetbuttcap%
\pgfsetmiterjoin%
\definecolor{currentfill}{rgb}{0.756863,0.988235,0.756863}%
\pgfsetfillcolor{currentfill}%
\pgfsetlinewidth{0.000000pt}%
\definecolor{currentstroke}{rgb}{0.000000,0.000000,0.000000}%
\pgfsetstrokecolor{currentstroke}%
\pgfsetstrokeopacity{0.000000}%
\pgfsetdash{}{0pt}%
\pgfpathmoveto{\pgfqpoint{4.833144in}{1.857203in}}%
\pgfpathlineto{\pgfqpoint{6.272000in}{1.857203in}}%
\pgfpathlineto{\pgfqpoint{6.272000in}{2.062797in}}%
\pgfpathlineto{\pgfqpoint{4.833144in}{2.062797in}}%
\pgfpathclose%
\pgfusepath{fill}%
\end{pgfscope}%
\begin{pgfscope}%
\pgfpathrectangle{\pgfqpoint{1.024000in}{0.490000in}}{\pgfqpoint{5.248000in}{2.940000in}}%
\pgfusepath{clip}%
\pgfsetbuttcap%
\pgfsetmiterjoin%
\definecolor{currentfill}{rgb}{0.756863,0.988235,0.756863}%
\pgfsetfillcolor{currentfill}%
\pgfsetlinewidth{0.000000pt}%
\definecolor{currentstroke}{rgb}{0.000000,0.000000,0.000000}%
\pgfsetstrokecolor{currentstroke}%
\pgfsetstrokeopacity{0.000000}%
\pgfsetdash{}{0pt}%
\pgfpathmoveto{\pgfqpoint{1.510570in}{2.268392in}}%
\pgfpathlineto{\pgfqpoint{6.272000in}{2.268392in}}%
\pgfpathlineto{\pgfqpoint{6.272000in}{2.473986in}}%
\pgfpathlineto{\pgfqpoint{1.510570in}{2.473986in}}%
\pgfpathclose%
\pgfusepath{fill}%
\end{pgfscope}%
\begin{pgfscope}%
\pgfpathrectangle{\pgfqpoint{1.024000in}{0.490000in}}{\pgfqpoint{5.248000in}{2.940000in}}%
\pgfusepath{clip}%
\pgfsetbuttcap%
\pgfsetmiterjoin%
\definecolor{currentfill}{rgb}{0.756863,0.988235,0.756863}%
\pgfsetfillcolor{currentfill}%
\pgfsetlinewidth{0.000000pt}%
\definecolor{currentstroke}{rgb}{0.000000,0.000000,0.000000}%
\pgfsetstrokecolor{currentstroke}%
\pgfsetstrokeopacity{0.000000}%
\pgfsetdash{}{0pt}%
\pgfpathmoveto{\pgfqpoint{3.790495in}{2.679580in}}%
\pgfpathlineto{\pgfqpoint{6.272000in}{2.679580in}}%
\pgfpathlineto{\pgfqpoint{6.272000in}{2.885175in}}%
\pgfpathlineto{\pgfqpoint{3.790495in}{2.885175in}}%
\pgfpathclose%
\pgfusepath{fill}%
\end{pgfscope}%
\begin{pgfscope}%
\pgfpathrectangle{\pgfqpoint{1.024000in}{0.490000in}}{\pgfqpoint{5.248000in}{2.940000in}}%
\pgfusepath{clip}%
\pgfsetbuttcap%
\pgfsetmiterjoin%
\definecolor{currentfill}{rgb}{0.756863,0.988235,0.756863}%
\pgfsetfillcolor{currentfill}%
\pgfsetlinewidth{0.000000pt}%
\definecolor{currentstroke}{rgb}{0.000000,0.000000,0.000000}%
\pgfsetstrokecolor{currentstroke}%
\pgfsetstrokeopacity{0.000000}%
\pgfsetdash{}{0pt}%
\pgfpathmoveto{\pgfqpoint{1.872021in}{3.090769in}}%
\pgfpathlineto{\pgfqpoint{6.272000in}{3.090769in}}%
\pgfpathlineto{\pgfqpoint{6.272000in}{3.296364in}}%
\pgfpathlineto{\pgfqpoint{1.872021in}{3.296364in}}%
\pgfpathclose%
\pgfusepath{fill}%
\end{pgfscope}%
\begin{pgfscope}%
\pgfsetbuttcap%
\pgfsetroundjoin%
\definecolor{currentfill}{rgb}{0.000000,0.000000,0.000000}%
\pgfsetfillcolor{currentfill}%
\pgfsetlinewidth{0.803000pt}%
\definecolor{currentstroke}{rgb}{0.000000,0.000000,0.000000}%
\pgfsetstrokecolor{currentstroke}%
\pgfsetdash{}{0pt}%
\pgfsys@defobject{currentmarker}{\pgfqpoint{0.000000in}{-0.048611in}}{\pgfqpoint{0.000000in}{0.000000in}}{%
\pgfpathmoveto{\pgfqpoint{0.000000in}{0.000000in}}%
\pgfpathlineto{\pgfqpoint{0.000000in}{-0.048611in}}%
\pgfusepath{stroke,fill}%
}%
\begin{pgfscope}%
\pgfsys@transformshift{2.073600in}{0.490000in}%
\pgfsys@useobject{currentmarker}{}%
\end{pgfscope}%
\end{pgfscope}%
\begin{pgfscope}%
\definecolor{textcolor}{rgb}{0.000000,0.000000,0.000000}%
\pgfsetstrokecolor{textcolor}%
\pgfsetfillcolor{textcolor}%
\pgftext[x=2.073600in,y=0.392778in,,top]{\color{textcolor}\rmfamily\fontsize{10.000000}{12.000000}\selectfont 0.2}%
\end{pgfscope}%
\begin{pgfscope}%
\pgfsetbuttcap%
\pgfsetroundjoin%
\definecolor{currentfill}{rgb}{0.000000,0.000000,0.000000}%
\pgfsetfillcolor{currentfill}%
\pgfsetlinewidth{0.803000pt}%
\definecolor{currentstroke}{rgb}{0.000000,0.000000,0.000000}%
\pgfsetstrokecolor{currentstroke}%
\pgfsetdash{}{0pt}%
\pgfsys@defobject{currentmarker}{\pgfqpoint{0.000000in}{-0.048611in}}{\pgfqpoint{0.000000in}{0.000000in}}{%
\pgfpathmoveto{\pgfqpoint{0.000000in}{0.000000in}}%
\pgfpathlineto{\pgfqpoint{0.000000in}{-0.048611in}}%
\pgfusepath{stroke,fill}%
}%
\begin{pgfscope}%
\pgfsys@transformshift{3.123200in}{0.490000in}%
\pgfsys@useobject{currentmarker}{}%
\end{pgfscope}%
\end{pgfscope}%
\begin{pgfscope}%
\definecolor{textcolor}{rgb}{0.000000,0.000000,0.000000}%
\pgfsetstrokecolor{textcolor}%
\pgfsetfillcolor{textcolor}%
\pgftext[x=3.123200in,y=0.392778in,,top]{\color{textcolor}\rmfamily\fontsize{10.000000}{12.000000}\selectfont 0.4}%
\end{pgfscope}%
\begin{pgfscope}%
\pgfsetbuttcap%
\pgfsetroundjoin%
\definecolor{currentfill}{rgb}{0.000000,0.000000,0.000000}%
\pgfsetfillcolor{currentfill}%
\pgfsetlinewidth{0.803000pt}%
\definecolor{currentstroke}{rgb}{0.000000,0.000000,0.000000}%
\pgfsetstrokecolor{currentstroke}%
\pgfsetdash{}{0pt}%
\pgfsys@defobject{currentmarker}{\pgfqpoint{0.000000in}{-0.048611in}}{\pgfqpoint{0.000000in}{0.000000in}}{%
\pgfpathmoveto{\pgfqpoint{0.000000in}{0.000000in}}%
\pgfpathlineto{\pgfqpoint{0.000000in}{-0.048611in}}%
\pgfusepath{stroke,fill}%
}%
\begin{pgfscope}%
\pgfsys@transformshift{4.172800in}{0.490000in}%
\pgfsys@useobject{currentmarker}{}%
\end{pgfscope}%
\end{pgfscope}%
\begin{pgfscope}%
\definecolor{textcolor}{rgb}{0.000000,0.000000,0.000000}%
\pgfsetstrokecolor{textcolor}%
\pgfsetfillcolor{textcolor}%
\pgftext[x=4.172800in,y=0.392778in,,top]{\color{textcolor}\rmfamily\fontsize{10.000000}{12.000000}\selectfont 0.6}%
\end{pgfscope}%
\begin{pgfscope}%
\pgfsetbuttcap%
\pgfsetroundjoin%
\definecolor{currentfill}{rgb}{0.000000,0.000000,0.000000}%
\pgfsetfillcolor{currentfill}%
\pgfsetlinewidth{0.803000pt}%
\definecolor{currentstroke}{rgb}{0.000000,0.000000,0.000000}%
\pgfsetstrokecolor{currentstroke}%
\pgfsetdash{}{0pt}%
\pgfsys@defobject{currentmarker}{\pgfqpoint{0.000000in}{-0.048611in}}{\pgfqpoint{0.000000in}{0.000000in}}{%
\pgfpathmoveto{\pgfqpoint{0.000000in}{0.000000in}}%
\pgfpathlineto{\pgfqpoint{0.000000in}{-0.048611in}}%
\pgfusepath{stroke,fill}%
}%
\begin{pgfscope}%
\pgfsys@transformshift{5.222400in}{0.490000in}%
\pgfsys@useobject{currentmarker}{}%
\end{pgfscope}%
\end{pgfscope}%
\begin{pgfscope}%
\definecolor{textcolor}{rgb}{0.000000,0.000000,0.000000}%
\pgfsetstrokecolor{textcolor}%
\pgfsetfillcolor{textcolor}%
\pgftext[x=5.222400in,y=0.392778in,,top]{\color{textcolor}\rmfamily\fontsize{10.000000}{12.000000}\selectfont 0.8}%
\end{pgfscope}%
\begin{pgfscope}%
\pgfsetbuttcap%
\pgfsetroundjoin%
\definecolor{currentfill}{rgb}{0.000000,0.000000,0.000000}%
\pgfsetfillcolor{currentfill}%
\pgfsetlinewidth{0.803000pt}%
\definecolor{currentstroke}{rgb}{0.000000,0.000000,0.000000}%
\pgfsetstrokecolor{currentstroke}%
\pgfsetdash{}{0pt}%
\pgfsys@defobject{currentmarker}{\pgfqpoint{0.000000in}{-0.048611in}}{\pgfqpoint{0.000000in}{0.000000in}}{%
\pgfpathmoveto{\pgfqpoint{0.000000in}{0.000000in}}%
\pgfpathlineto{\pgfqpoint{0.000000in}{-0.048611in}}%
\pgfusepath{stroke,fill}%
}%
\begin{pgfscope}%
\pgfsys@transformshift{6.272000in}{0.490000in}%
\pgfsys@useobject{currentmarker}{}%
\end{pgfscope}%
\end{pgfscope}%
\begin{pgfscope}%
\definecolor{textcolor}{rgb}{0.000000,0.000000,0.000000}%
\pgfsetstrokecolor{textcolor}%
\pgfsetfillcolor{textcolor}%
\pgftext[x=6.272000in,y=0.392778in,,top]{\color{textcolor}\rmfamily\fontsize{10.000000}{12.000000}\selectfont 1.0}%
\end{pgfscope}%
\begin{pgfscope}%
\definecolor{textcolor}{rgb}{0.000000,0.000000,0.000000}%
\pgfsetstrokecolor{textcolor}%
\pgfsetfillcolor{textcolor}%
\pgftext[x=3.648000in,y=0.207778in,,top]{\color{textcolor}\rmfamily\fontsize{10.000000}{12.000000}\selectfont Percentage}%
\end{pgfscope}%
\begin{pgfscope}%
\pgfsetbuttcap%
\pgfsetroundjoin%
\definecolor{currentfill}{rgb}{0.000000,0.000000,0.000000}%
\pgfsetfillcolor{currentfill}%
\pgfsetlinewidth{0.803000pt}%
\definecolor{currentstroke}{rgb}{0.000000,0.000000,0.000000}%
\pgfsetstrokecolor{currentstroke}%
\pgfsetdash{}{0pt}%
\pgfsys@defobject{currentmarker}{\pgfqpoint{-0.048611in}{0.000000in}}{\pgfqpoint{-0.000000in}{0.000000in}}{%
\pgfpathmoveto{\pgfqpoint{-0.000000in}{0.000000in}}%
\pgfpathlineto{\pgfqpoint{-0.048611in}{0.000000in}}%
\pgfusepath{stroke,fill}%
}%
\begin{pgfscope}%
\pgfsys@transformshift{1.024000in}{0.726434in}%
\pgfsys@useobject{currentmarker}{}%
\end{pgfscope}%
\end{pgfscope}%
\begin{pgfscope}%
\definecolor{textcolor}{rgb}{0.000000,0.000000,0.000000}%
\pgfsetstrokecolor{textcolor}%
\pgfsetfillcolor{textcolor}%
\pgftext[x=0.542278in, y=0.668267in, left, base]{\color{textcolor}\rmfamily\fontsize{12.000000}{14.400000}\selectfont qtype}%
\end{pgfscope}%
\begin{pgfscope}%
\pgfsetbuttcap%
\pgfsetroundjoin%
\definecolor{currentfill}{rgb}{0.000000,0.000000,0.000000}%
\pgfsetfillcolor{currentfill}%
\pgfsetlinewidth{0.803000pt}%
\definecolor{currentstroke}{rgb}{0.000000,0.000000,0.000000}%
\pgfsetstrokecolor{currentstroke}%
\pgfsetdash{}{0pt}%
\pgfsys@defobject{currentmarker}{\pgfqpoint{-0.048611in}{0.000000in}}{\pgfqpoint{-0.000000in}{0.000000in}}{%
\pgfpathmoveto{\pgfqpoint{-0.000000in}{0.000000in}}%
\pgfpathlineto{\pgfqpoint{-0.048611in}{0.000000in}}%
\pgfusepath{stroke,fill}%
}%
\begin{pgfscope}%
\pgfsys@transformshift{1.024000in}{1.137622in}%
\pgfsys@useobject{currentmarker}{}%
\end{pgfscope}%
\end{pgfscope}%
\begin{pgfscope}%
\definecolor{textcolor}{rgb}{0.000000,0.000000,0.000000}%
\pgfsetstrokecolor{textcolor}%
\pgfsetfillcolor{textcolor}%
\pgftext[x=0.058111in, y=1.079456in, left, base]{\color{textcolor}\rmfamily\fontsize{12.000000}{14.400000}\selectfont around\_topx}%
\end{pgfscope}%
\begin{pgfscope}%
\pgfsetbuttcap%
\pgfsetroundjoin%
\definecolor{currentfill}{rgb}{0.000000,0.000000,0.000000}%
\pgfsetfillcolor{currentfill}%
\pgfsetlinewidth{0.803000pt}%
\definecolor{currentstroke}{rgb}{0.000000,0.000000,0.000000}%
\pgfsetstrokecolor{currentstroke}%
\pgfsetdash{}{0pt}%
\pgfsys@defobject{currentmarker}{\pgfqpoint{-0.048611in}{0.000000in}}{\pgfqpoint{-0.000000in}{0.000000in}}{%
\pgfpathmoveto{\pgfqpoint{-0.000000in}{0.000000in}}%
\pgfpathlineto{\pgfqpoint{-0.048611in}{0.000000in}}%
\pgfusepath{stroke,fill}%
}%
\begin{pgfscope}%
\pgfsys@transformshift{1.024000in}{1.548811in}%
\pgfsys@useobject{currentmarker}{}%
\end{pgfscope}%
\end{pgfscope}%
\begin{pgfscope}%
\definecolor{textcolor}{rgb}{0.000000,0.000000,0.000000}%
\pgfsetstrokecolor{textcolor}%
\pgfsetfillcolor{textcolor}%
\pgftext[x=0.390111in, y=1.490645in, left, base]{\color{textcolor}\rmfamily\fontsize{12.000000}{14.400000}\selectfont maxdist}%
\end{pgfscope}%
\begin{pgfscope}%
\pgfsetbuttcap%
\pgfsetroundjoin%
\definecolor{currentfill}{rgb}{0.000000,0.000000,0.000000}%
\pgfsetfillcolor{currentfill}%
\pgfsetlinewidth{0.803000pt}%
\definecolor{currentstroke}{rgb}{0.000000,0.000000,0.000000}%
\pgfsetstrokecolor{currentstroke}%
\pgfsetdash{}{0pt}%
\pgfsys@defobject{currentmarker}{\pgfqpoint{-0.048611in}{0.000000in}}{\pgfqpoint{-0.000000in}{0.000000in}}{%
\pgfpathmoveto{\pgfqpoint{-0.000000in}{0.000000in}}%
\pgfpathlineto{\pgfqpoint{-0.048611in}{0.000000in}}%
\pgfusepath{stroke,fill}%
}%
\begin{pgfscope}%
\pgfsys@transformshift{1.024000in}{1.960000in}%
\pgfsys@useobject{currentmarker}{}%
\end{pgfscope}%
\end{pgfscope}%
\begin{pgfscope}%
\definecolor{textcolor}{rgb}{0.000000,0.000000,0.000000}%
\pgfsetstrokecolor{textcolor}%
\pgfsetfillcolor{textcolor}%
\pgftext[x=0.167611in, y=1.902167in, left, base]{\color{textcolor}\rmfamily\fontsize{12.000000}{14.400000}\selectfont target\_nwr}%
\end{pgfscope}%
\begin{pgfscope}%
\pgfsetbuttcap%
\pgfsetroundjoin%
\definecolor{currentfill}{rgb}{0.000000,0.000000,0.000000}%
\pgfsetfillcolor{currentfill}%
\pgfsetlinewidth{0.803000pt}%
\definecolor{currentstroke}{rgb}{0.000000,0.000000,0.000000}%
\pgfsetstrokecolor{currentstroke}%
\pgfsetdash{}{0pt}%
\pgfsys@defobject{currentmarker}{\pgfqpoint{-0.048611in}{0.000000in}}{\pgfqpoint{-0.000000in}{0.000000in}}{%
\pgfpathmoveto{\pgfqpoint{-0.000000in}{0.000000in}}%
\pgfpathlineto{\pgfqpoint{-0.048611in}{0.000000in}}%
\pgfusepath{stroke,fill}%
}%
\begin{pgfscope}%
\pgfsys@transformshift{1.024000in}{2.371189in}%
\pgfsys@useobject{currentmarker}{}%
\end{pgfscope}%
\end{pgfscope}%
\begin{pgfscope}%
\definecolor{textcolor}{rgb}{0.000000,0.000000,0.000000}%
\pgfsetstrokecolor{textcolor}%
\pgfsetfillcolor{textcolor}%
\pgftext[x=0.143611in, y=2.313022in, left, base]{\color{textcolor}\rmfamily\fontsize{12.000000}{14.400000}\selectfont center\_nwr}%
\end{pgfscope}%
\begin{pgfscope}%
\pgfsetbuttcap%
\pgfsetroundjoin%
\definecolor{currentfill}{rgb}{0.000000,0.000000,0.000000}%
\pgfsetfillcolor{currentfill}%
\pgfsetlinewidth{0.803000pt}%
\definecolor{currentstroke}{rgb}{0.000000,0.000000,0.000000}%
\pgfsetstrokecolor{currentstroke}%
\pgfsetdash{}{0pt}%
\pgfsys@defobject{currentmarker}{\pgfqpoint{-0.048611in}{0.000000in}}{\pgfqpoint{-0.000000in}{0.000000in}}{%
\pgfpathmoveto{\pgfqpoint{-0.000000in}{0.000000in}}%
\pgfpathlineto{\pgfqpoint{-0.048611in}{0.000000in}}%
\pgfusepath{stroke,fill}%
}%
\begin{pgfscope}%
\pgfsys@transformshift{1.024000in}{2.782378in}%
\pgfsys@useobject{currentmarker}{}%
\end{pgfscope}%
\end{pgfscope}%
\begin{pgfscope}%
\definecolor{textcolor}{rgb}{0.000000,0.000000,0.000000}%
\pgfsetstrokecolor{textcolor}%
\pgfsetfillcolor{textcolor}%
\pgftext[x=0.639278in, y=2.724211in, left, base]{\color{textcolor}\rmfamily\fontsize{12.000000}{14.400000}\selectfont area}%
\end{pgfscope}%
\begin{pgfscope}%
\pgfsetbuttcap%
\pgfsetroundjoin%
\definecolor{currentfill}{rgb}{0.000000,0.000000,0.000000}%
\pgfsetfillcolor{currentfill}%
\pgfsetlinewidth{0.803000pt}%
\definecolor{currentstroke}{rgb}{0.000000,0.000000,0.000000}%
\pgfsetstrokecolor{currentstroke}%
\pgfsetdash{}{0pt}%
\pgfsys@defobject{currentmarker}{\pgfqpoint{-0.048611in}{0.000000in}}{\pgfqpoint{-0.000000in}{0.000000in}}{%
\pgfpathmoveto{\pgfqpoint{-0.000000in}{0.000000in}}%
\pgfpathlineto{\pgfqpoint{-0.048611in}{0.000000in}}%
\pgfusepath{stroke,fill}%
}%
\begin{pgfscope}%
\pgfsys@transformshift{1.024000in}{3.193566in}%
\pgfsys@useobject{currentmarker}{}%
\end{pgfscope}%
\end{pgfscope}%
\begin{pgfscope}%
\definecolor{textcolor}{rgb}{0.000000,0.000000,0.000000}%
\pgfsetstrokecolor{textcolor}%
\pgfsetfillcolor{textcolor}%
\pgftext[x=0.310944in, y=3.135400in, left, base]{\color{textcolor}\rmfamily\fontsize{12.000000}{14.400000}\selectfont structure}%
\end{pgfscope}%
\begin{pgfscope}%
\pgfsetrectcap%
\pgfsetmiterjoin%
\pgfsetlinewidth{0.803000pt}%
\definecolor{currentstroke}{rgb}{0.000000,0.000000,0.000000}%
\pgfsetstrokecolor{currentstroke}%
\pgfsetdash{}{0pt}%
\pgfpathmoveto{\pgfqpoint{1.024000in}{0.490000in}}%
\pgfpathlineto{\pgfqpoint{1.024000in}{3.430000in}}%
\pgfusepath{stroke}%
\end{pgfscope}%
\begin{pgfscope}%
\pgfsetrectcap%
\pgfsetmiterjoin%
\pgfsetlinewidth{0.803000pt}%
\definecolor{currentstroke}{rgb}{0.000000,0.000000,0.000000}%
\pgfsetstrokecolor{currentstroke}%
\pgfsetdash{}{0pt}%
\pgfpathmoveto{\pgfqpoint{6.272000in}{0.490000in}}%
\pgfpathlineto{\pgfqpoint{6.272000in}{3.430000in}}%
\pgfusepath{stroke}%
\end{pgfscope}%
\begin{pgfscope}%
\pgfsetrectcap%
\pgfsetmiterjoin%
\pgfsetlinewidth{0.803000pt}%
\definecolor{currentstroke}{rgb}{0.000000,0.000000,0.000000}%
\pgfsetstrokecolor{currentstroke}%
\pgfsetdash{}{0pt}%
\pgfpathmoveto{\pgfqpoint{1.024000in}{0.490000in}}%
\pgfpathlineto{\pgfqpoint{6.272000in}{0.490000in}}%
\pgfusepath{stroke}%
\end{pgfscope}%
\begin{pgfscope}%
\pgfsetrectcap%
\pgfsetmiterjoin%
\pgfsetlinewidth{0.803000pt}%
\definecolor{currentstroke}{rgb}{0.000000,0.000000,0.000000}%
\pgfsetstrokecolor{currentstroke}%
\pgfsetdash{}{0pt}%
\pgfpathmoveto{\pgfqpoint{1.024000in}{3.430000in}}%
\pgfpathlineto{\pgfqpoint{6.272000in}{3.430000in}}%
\pgfusepath{stroke}%
\end{pgfscope}%
\end{pgfpicture}%
\makeatother%
\endgroup%
}
  \caption[Errors pre-trained on \nlmtwoone{}]{Percentage of queries with an
    error in a specific operator. The model was pre-trained on \nlmapstwoone{}
    and tested on \nlmapsfour{}.}
  \label{fig:v21-errors}
\end{figure}

The models trained on the purely synthetic training sets of \nlmthreea{} and
\nlmthreeb{} achieve an extremely high accuracy of around \SI{98}{\%} on the
test set of \nlmthreea{}. The model trained on the non-noisy \nlmthreea{} turns
out to be less robust on the noisy \nlmthreeb{} test set with a performance drop
of almost \SI{20}{\%} whereas the accuracy of the model trained on \nlmthreeb{}
only drops by around \SI{9}{\%}. More interestingly, the \nlmthreeb{} model
significantly outperforms the \nlmthreea{} model on the \nlmtwoone{} test set,
as well, which suggests that the added noise serves as a means of regularization
and avoids overfitting on the synthetic queries. When confronted with the real
queries in \nlmapsfour{}, both models’ accuracy again drops starkly to around
\SI{23}{\%} – with the noisy model still performing slightly better than its
non-noisy counterpart. As shown in Figure~\ref{fig:v3b-errors}, the improvement
with respect to the model trained on \nlmtwoone{} is mostly due to the greater
variety of location names (cf. Section~\ref{sec:little-location-variety}), which
significantly reduces the number of errors in the \mrl{area} and
\mrl{center_nwr} operators. The main source of errors remains choosing the
correct OSM tags in the \mrl{target_nwr} operator.

\begin{figure}[h]
  \centering
  \resizebox{\textwidth}{!}{\input{fig/n3b_n4_errors.pgf}}
  \caption[Errors pre-trained on \nlmtwoone{}]{Percentage of queries with an
    error in a specific operator. The model was pre-trained on \nlmapsthreeb{}
    and tested on \nlmapsfour{}.}
  \label{fig:v3b-errors}
\end{figure}

Combining \nlmtwoone{} and the new synthetic data from \nlmthreea{} or
\nlmthreeb{} yields further improvement as evidenced by the performance of the
models trained on \nlmthreenormal{} and \nlmthree{}. They perform well on
\nlmtwoone{} as well as on the variations of \nlmthree{}, but the accuracy on
the noisy versions is still noticeably smaller than on their non-noisy
counterparts. While the comparison of the two models trained on \nlmthreea{} and
\nlmthreeb{} showed a clearly superior performance of the model trained on the
noisy data in all variations of \nlmtwo{} and \nlmthree{} except \nlmthreea{},
this is not observed when comparing the two models trained on \nlmthreenormal{}
and \nlmthree{}. Instead, they have a similar accuracy on all test sets with the
model trained on \nlmthreenormal{} being slightly superior on the non-noisy
datasets and the model trained on \nlmthree{} being slightly superior on the
noisy ones. Testing on the real user queries from \nlmfour{} reveals that the
model trained on the non-noisy data actually performs better in the end.
However, since \nlmfour{} was not available during the pre-training phase, at
that point the two models seemed comparable and our assumption was that the
model trained on \nlmthreeb{} would prove more robust against typing errors.

\begin{figure}[h]
  \centering
  \resizebox{\textwidth}{!}{\input{fig/n3_n4_errors.pgf}}
  \caption[Errors pre-trained on \nlmtwoone{}]{Percentage of queries with an
    error in a specific operator. The model was pre-trained on \nlmapsthree{}
    and tested on \nlmapsfour{}.}
  \label{fig:v3-errors}
\end{figure}

Figure~\ref{fig:pre-trained-learning-curve} shows the models’ learning curves on
the development set of \nlmthree{} during pre-training. While all models
converge fairly quickly, it can be seen that the model trained on \nlmtwoone{}
converges somewhat faster than the other models. This might again be due to the
simpler location names, which can be memorized by the model, while the other
models have to learn to copy arbitrary location names – including some that are
very long or contain rare characters.

\begin{figure}[h]
  \centering
  \resizebox{\textwidth}{!}{%% Creator: Matplotlib, PGF backend
%%
%% To include the figure in your LaTeX document, write
%%   \input{<filename>.pgf}
%%
%% Make sure the required packages are loaded in your preamble
%%   \usepackage{pgf}
%%
%% and, on pdftex
%%   \usepackage[utf8]{inputenc}\DeclareUnicodeCharacter{2212}{-}
%%
%% or, on luatex and xetex
%%   \usepackage{unicode-math}
%%
%% Figures using additional raster images can only be included by \input if
%% they are in the same directory as the main LaTeX file. For loading figures
%% from other directories you can use the `import` package
%%   \usepackage{import}
%%
%% and then include the figures with
%%   \import{<path to file>}{<filename>.pgf}
%%
%% Matplotlib used the following preamble
%%   \usepackage{fontspec}
%%   \setmainfont{LinLibertine_R.otf}[Path=/usr/share/texmf-dist/fonts/opentype/public/libertine/]
%%   \setsansfont{LinLibertine_R.otf}[Path=/usr/share/texmf-dist/fonts/opentype/public/libertine/]
%%   \setmonofont{DejaVuSansMono.ttf}[Path=/home/gorgor/.virtualenvs/ma/lib/python3.9/site-packages/matplotlib/mpl-data/fonts/ttf/]
%%
\begingroup%
\makeatletter%
\begin{pgfpicture}%
\pgfpathrectangle{\pgfpointorigin}{\pgfqpoint{6.400000in}{4.800000in}}%
\pgfusepath{use as bounding box, clip}%
\begin{pgfscope}%
\pgfsetbuttcap%
\pgfsetmiterjoin%
\definecolor{currentfill}{rgb}{1.000000,1.000000,1.000000}%
\pgfsetfillcolor{currentfill}%
\pgfsetlinewidth{0.000000pt}%
\definecolor{currentstroke}{rgb}{1.000000,1.000000,1.000000}%
\pgfsetstrokecolor{currentstroke}%
\pgfsetdash{}{0pt}%
\pgfpathmoveto{\pgfqpoint{0.000000in}{0.000000in}}%
\pgfpathlineto{\pgfqpoint{6.400000in}{0.000000in}}%
\pgfpathlineto{\pgfqpoint{6.400000in}{4.800000in}}%
\pgfpathlineto{\pgfqpoint{0.000000in}{4.800000in}}%
\pgfpathclose%
\pgfusepath{fill}%
\end{pgfscope}%
\begin{pgfscope}%
\pgfsetbuttcap%
\pgfsetmiterjoin%
\definecolor{currentfill}{rgb}{1.000000,1.000000,1.000000}%
\pgfsetfillcolor{currentfill}%
\pgfsetlinewidth{0.000000pt}%
\definecolor{currentstroke}{rgb}{0.000000,0.000000,0.000000}%
\pgfsetstrokecolor{currentstroke}%
\pgfsetstrokeopacity{0.000000}%
\pgfsetdash{}{0pt}%
\pgfpathmoveto{\pgfqpoint{0.561389in}{0.561666in}}%
\pgfpathlineto{\pgfqpoint{6.250000in}{0.561666in}}%
\pgfpathlineto{\pgfqpoint{6.250000in}{4.650000in}}%
\pgfpathlineto{\pgfqpoint{0.561389in}{4.650000in}}%
\pgfpathclose%
\pgfusepath{fill}%
\end{pgfscope}%
\begin{pgfscope}%
\pgfsetbuttcap%
\pgfsetroundjoin%
\definecolor{currentfill}{rgb}{0.000000,0.000000,0.000000}%
\pgfsetfillcolor{currentfill}%
\pgfsetlinewidth{0.803000pt}%
\definecolor{currentstroke}{rgb}{0.000000,0.000000,0.000000}%
\pgfsetstrokecolor{currentstroke}%
\pgfsetdash{}{0pt}%
\pgfsys@defobject{currentmarker}{\pgfqpoint{0.000000in}{-0.048611in}}{\pgfqpoint{0.000000in}{0.000000in}}{%
\pgfpathmoveto{\pgfqpoint{0.000000in}{0.000000in}}%
\pgfpathlineto{\pgfqpoint{0.000000in}{-0.048611in}}%
\pgfusepath{stroke,fill}%
}%
\begin{pgfscope}%
\pgfsys@transformshift{0.702429in}{0.561666in}%
\pgfsys@useobject{currentmarker}{}%
\end{pgfscope}%
\end{pgfscope}%
\begin{pgfscope}%
\definecolor{textcolor}{rgb}{0.000000,0.000000,0.000000}%
\pgfsetstrokecolor{textcolor}%
\pgfsetfillcolor{textcolor}%
\pgftext[x=0.702429in,y=0.464444in,,top]{\color{textcolor}\rmfamily\fontsize{10.000000}{12.000000}\selectfont 0}%
\end{pgfscope}%
\begin{pgfscope}%
\pgfsetbuttcap%
\pgfsetroundjoin%
\definecolor{currentfill}{rgb}{0.000000,0.000000,0.000000}%
\pgfsetfillcolor{currentfill}%
\pgfsetlinewidth{0.803000pt}%
\definecolor{currentstroke}{rgb}{0.000000,0.000000,0.000000}%
\pgfsetstrokecolor{currentstroke}%
\pgfsetdash{}{0pt}%
\pgfsys@defobject{currentmarker}{\pgfqpoint{0.000000in}{-0.048611in}}{\pgfqpoint{0.000000in}{0.000000in}}{%
\pgfpathmoveto{\pgfqpoint{0.000000in}{0.000000in}}%
\pgfpathlineto{\pgfqpoint{0.000000in}{-0.048611in}}%
\pgfusepath{stroke,fill}%
}%
\begin{pgfscope}%
\pgfsys@transformshift{1.877761in}{0.561666in}%
\pgfsys@useobject{currentmarker}{}%
\end{pgfscope}%
\end{pgfscope}%
\begin{pgfscope}%
\definecolor{textcolor}{rgb}{0.000000,0.000000,0.000000}%
\pgfsetstrokecolor{textcolor}%
\pgfsetfillcolor{textcolor}%
\pgftext[x=1.877761in,y=0.464444in,,top]{\color{textcolor}\rmfamily\fontsize{10.000000}{12.000000}\selectfont 20000}%
\end{pgfscope}%
\begin{pgfscope}%
\pgfsetbuttcap%
\pgfsetroundjoin%
\definecolor{currentfill}{rgb}{0.000000,0.000000,0.000000}%
\pgfsetfillcolor{currentfill}%
\pgfsetlinewidth{0.803000pt}%
\definecolor{currentstroke}{rgb}{0.000000,0.000000,0.000000}%
\pgfsetstrokecolor{currentstroke}%
\pgfsetdash{}{0pt}%
\pgfsys@defobject{currentmarker}{\pgfqpoint{0.000000in}{-0.048611in}}{\pgfqpoint{0.000000in}{0.000000in}}{%
\pgfpathmoveto{\pgfqpoint{0.000000in}{0.000000in}}%
\pgfpathlineto{\pgfqpoint{0.000000in}{-0.048611in}}%
\pgfusepath{stroke,fill}%
}%
\begin{pgfscope}%
\pgfsys@transformshift{3.053094in}{0.561666in}%
\pgfsys@useobject{currentmarker}{}%
\end{pgfscope}%
\end{pgfscope}%
\begin{pgfscope}%
\definecolor{textcolor}{rgb}{0.000000,0.000000,0.000000}%
\pgfsetstrokecolor{textcolor}%
\pgfsetfillcolor{textcolor}%
\pgftext[x=3.053094in,y=0.464444in,,top]{\color{textcolor}\rmfamily\fontsize{10.000000}{12.000000}\selectfont 40000}%
\end{pgfscope}%
\begin{pgfscope}%
\pgfsetbuttcap%
\pgfsetroundjoin%
\definecolor{currentfill}{rgb}{0.000000,0.000000,0.000000}%
\pgfsetfillcolor{currentfill}%
\pgfsetlinewidth{0.803000pt}%
\definecolor{currentstroke}{rgb}{0.000000,0.000000,0.000000}%
\pgfsetstrokecolor{currentstroke}%
\pgfsetdash{}{0pt}%
\pgfsys@defobject{currentmarker}{\pgfqpoint{0.000000in}{-0.048611in}}{\pgfqpoint{0.000000in}{0.000000in}}{%
\pgfpathmoveto{\pgfqpoint{0.000000in}{0.000000in}}%
\pgfpathlineto{\pgfqpoint{0.000000in}{-0.048611in}}%
\pgfusepath{stroke,fill}%
}%
\begin{pgfscope}%
\pgfsys@transformshift{4.228427in}{0.561666in}%
\pgfsys@useobject{currentmarker}{}%
\end{pgfscope}%
\end{pgfscope}%
\begin{pgfscope}%
\definecolor{textcolor}{rgb}{0.000000,0.000000,0.000000}%
\pgfsetstrokecolor{textcolor}%
\pgfsetfillcolor{textcolor}%
\pgftext[x=4.228427in,y=0.464444in,,top]{\color{textcolor}\rmfamily\fontsize{10.000000}{12.000000}\selectfont 60000}%
\end{pgfscope}%
\begin{pgfscope}%
\pgfsetbuttcap%
\pgfsetroundjoin%
\definecolor{currentfill}{rgb}{0.000000,0.000000,0.000000}%
\pgfsetfillcolor{currentfill}%
\pgfsetlinewidth{0.803000pt}%
\definecolor{currentstroke}{rgb}{0.000000,0.000000,0.000000}%
\pgfsetstrokecolor{currentstroke}%
\pgfsetdash{}{0pt}%
\pgfsys@defobject{currentmarker}{\pgfqpoint{0.000000in}{-0.048611in}}{\pgfqpoint{0.000000in}{0.000000in}}{%
\pgfpathmoveto{\pgfqpoint{0.000000in}{0.000000in}}%
\pgfpathlineto{\pgfqpoint{0.000000in}{-0.048611in}}%
\pgfusepath{stroke,fill}%
}%
\begin{pgfscope}%
\pgfsys@transformshift{5.403760in}{0.561666in}%
\pgfsys@useobject{currentmarker}{}%
\end{pgfscope}%
\end{pgfscope}%
\begin{pgfscope}%
\definecolor{textcolor}{rgb}{0.000000,0.000000,0.000000}%
\pgfsetstrokecolor{textcolor}%
\pgfsetfillcolor{textcolor}%
\pgftext[x=5.403760in,y=0.464444in,,top]{\color{textcolor}\rmfamily\fontsize{10.000000}{12.000000}\selectfont 80000}%
\end{pgfscope}%
\begin{pgfscope}%
\definecolor{textcolor}{rgb}{0.000000,0.000000,0.000000}%
\pgfsetstrokecolor{textcolor}%
\pgfsetfillcolor{textcolor}%
\pgftext[x=3.405694in,y=0.279444in,,top]{\color{textcolor}\rmfamily\fontsize{10.000000}{12.000000}\selectfont Steps}%
\end{pgfscope}%
\begin{pgfscope}%
\pgfsetbuttcap%
\pgfsetroundjoin%
\definecolor{currentfill}{rgb}{0.000000,0.000000,0.000000}%
\pgfsetfillcolor{currentfill}%
\pgfsetlinewidth{0.803000pt}%
\definecolor{currentstroke}{rgb}{0.000000,0.000000,0.000000}%
\pgfsetstrokecolor{currentstroke}%
\pgfsetdash{}{0pt}%
\pgfsys@defobject{currentmarker}{\pgfqpoint{-0.048611in}{0.000000in}}{\pgfqpoint{-0.000000in}{0.000000in}}{%
\pgfpathmoveto{\pgfqpoint{-0.000000in}{0.000000in}}%
\pgfpathlineto{\pgfqpoint{-0.048611in}{0.000000in}}%
\pgfusepath{stroke,fill}%
}%
\begin{pgfscope}%
\pgfsys@transformshift{0.561389in}{0.747499in}%
\pgfsys@useobject{currentmarker}{}%
\end{pgfscope}%
\end{pgfscope}%
\begin{pgfscope}%
\definecolor{textcolor}{rgb}{0.000000,0.000000,0.000000}%
\pgfsetstrokecolor{textcolor}%
\pgfsetfillcolor{textcolor}%
\pgftext[x=0.399583in, y=0.699027in, left, base]{\color{textcolor}\rmfamily\fontsize{10.000000}{12.000000}\selectfont 0}%
\end{pgfscope}%
\begin{pgfscope}%
\pgfsetbuttcap%
\pgfsetroundjoin%
\definecolor{currentfill}{rgb}{0.000000,0.000000,0.000000}%
\pgfsetfillcolor{currentfill}%
\pgfsetlinewidth{0.803000pt}%
\definecolor{currentstroke}{rgb}{0.000000,0.000000,0.000000}%
\pgfsetstrokecolor{currentstroke}%
\pgfsetdash{}{0pt}%
\pgfsys@defobject{currentmarker}{\pgfqpoint{-0.048611in}{0.000000in}}{\pgfqpoint{-0.000000in}{0.000000in}}{%
\pgfpathmoveto{\pgfqpoint{-0.000000in}{0.000000in}}%
\pgfpathlineto{\pgfqpoint{-0.048611in}{0.000000in}}%
\pgfusepath{stroke,fill}%
}%
\begin{pgfscope}%
\pgfsys@transformshift{0.561389in}{1.590495in}%
\pgfsys@useobject{currentmarker}{}%
\end{pgfscope}%
\end{pgfscope}%
\begin{pgfscope}%
\definecolor{textcolor}{rgb}{0.000000,0.000000,0.000000}%
\pgfsetstrokecolor{textcolor}%
\pgfsetfillcolor{textcolor}%
\pgftext[x=0.335000in, y=1.542023in, left, base]{\color{textcolor}\rmfamily\fontsize{10.000000}{12.000000}\selectfont 20}%
\end{pgfscope}%
\begin{pgfscope}%
\pgfsetbuttcap%
\pgfsetroundjoin%
\definecolor{currentfill}{rgb}{0.000000,0.000000,0.000000}%
\pgfsetfillcolor{currentfill}%
\pgfsetlinewidth{0.803000pt}%
\definecolor{currentstroke}{rgb}{0.000000,0.000000,0.000000}%
\pgfsetstrokecolor{currentstroke}%
\pgfsetdash{}{0pt}%
\pgfsys@defobject{currentmarker}{\pgfqpoint{-0.048611in}{0.000000in}}{\pgfqpoint{-0.000000in}{0.000000in}}{%
\pgfpathmoveto{\pgfqpoint{-0.000000in}{0.000000in}}%
\pgfpathlineto{\pgfqpoint{-0.048611in}{0.000000in}}%
\pgfusepath{stroke,fill}%
}%
\begin{pgfscope}%
\pgfsys@transformshift{0.561389in}{2.433491in}%
\pgfsys@useobject{currentmarker}{}%
\end{pgfscope}%
\end{pgfscope}%
\begin{pgfscope}%
\definecolor{textcolor}{rgb}{0.000000,0.000000,0.000000}%
\pgfsetstrokecolor{textcolor}%
\pgfsetfillcolor{textcolor}%
\pgftext[x=0.335000in, y=2.385018in, left, base]{\color{textcolor}\rmfamily\fontsize{10.000000}{12.000000}\selectfont 40}%
\end{pgfscope}%
\begin{pgfscope}%
\pgfsetbuttcap%
\pgfsetroundjoin%
\definecolor{currentfill}{rgb}{0.000000,0.000000,0.000000}%
\pgfsetfillcolor{currentfill}%
\pgfsetlinewidth{0.803000pt}%
\definecolor{currentstroke}{rgb}{0.000000,0.000000,0.000000}%
\pgfsetstrokecolor{currentstroke}%
\pgfsetdash{}{0pt}%
\pgfsys@defobject{currentmarker}{\pgfqpoint{-0.048611in}{0.000000in}}{\pgfqpoint{-0.000000in}{0.000000in}}{%
\pgfpathmoveto{\pgfqpoint{-0.000000in}{0.000000in}}%
\pgfpathlineto{\pgfqpoint{-0.048611in}{0.000000in}}%
\pgfusepath{stroke,fill}%
}%
\begin{pgfscope}%
\pgfsys@transformshift{0.561389in}{3.276486in}%
\pgfsys@useobject{currentmarker}{}%
\end{pgfscope}%
\end{pgfscope}%
\begin{pgfscope}%
\definecolor{textcolor}{rgb}{0.000000,0.000000,0.000000}%
\pgfsetstrokecolor{textcolor}%
\pgfsetfillcolor{textcolor}%
\pgftext[x=0.335000in, y=3.228014in, left, base]{\color{textcolor}\rmfamily\fontsize{10.000000}{12.000000}\selectfont 60}%
\end{pgfscope}%
\begin{pgfscope}%
\pgfsetbuttcap%
\pgfsetroundjoin%
\definecolor{currentfill}{rgb}{0.000000,0.000000,0.000000}%
\pgfsetfillcolor{currentfill}%
\pgfsetlinewidth{0.803000pt}%
\definecolor{currentstroke}{rgb}{0.000000,0.000000,0.000000}%
\pgfsetstrokecolor{currentstroke}%
\pgfsetdash{}{0pt}%
\pgfsys@defobject{currentmarker}{\pgfqpoint{-0.048611in}{0.000000in}}{\pgfqpoint{-0.000000in}{0.000000in}}{%
\pgfpathmoveto{\pgfqpoint{-0.000000in}{0.000000in}}%
\pgfpathlineto{\pgfqpoint{-0.048611in}{0.000000in}}%
\pgfusepath{stroke,fill}%
}%
\begin{pgfscope}%
\pgfsys@transformshift{0.561389in}{4.119482in}%
\pgfsys@useobject{currentmarker}{}%
\end{pgfscope}%
\end{pgfscope}%
\begin{pgfscope}%
\definecolor{textcolor}{rgb}{0.000000,0.000000,0.000000}%
\pgfsetstrokecolor{textcolor}%
\pgfsetfillcolor{textcolor}%
\pgftext[x=0.335000in, y=4.071010in, left, base]{\color{textcolor}\rmfamily\fontsize{10.000000}{12.000000}\selectfont 80}%
\end{pgfscope}%
\begin{pgfscope}%
\definecolor{textcolor}{rgb}{0.000000,0.000000,0.000000}%
\pgfsetstrokecolor{textcolor}%
\pgfsetfillcolor{textcolor}%
\pgftext[x=0.279444in,y=2.605833in,,bottom,rotate=90.000000]{\color{textcolor}\rmfamily\fontsize{10.000000}{12.000000}\selectfont NLMaps\,v3 Dev Accuracy}%
\end{pgfscope}%
\begin{pgfscope}%
\pgfpathrectangle{\pgfqpoint{0.561389in}{0.561666in}}{\pgfqpoint{5.688611in}{4.088334in}}%
\pgfusepath{clip}%
\pgfsetrectcap%
\pgfsetroundjoin%
\pgfsetlinewidth{1.505625pt}%
\definecolor{currentstroke}{rgb}{0.121569,0.466667,0.705882}%
\pgfsetstrokecolor{currentstroke}%
\pgfsetdash{}{0pt}%
\pgfpathmoveto{\pgfqpoint{0.819962in}{1.062286in}}%
\pgfpathlineto{\pgfqpoint{0.937495in}{2.288616in}}%
\pgfpathlineto{\pgfqpoint{1.055028in}{2.569371in}}%
\pgfpathlineto{\pgfqpoint{1.172562in}{2.575449in}}%
\pgfpathlineto{\pgfqpoint{1.290095in}{2.580310in}}%
\pgfpathlineto{\pgfqpoint{1.407628in}{2.652018in}}%
\pgfpathlineto{\pgfqpoint{1.525162in}{2.690911in}}%
\pgfpathlineto{\pgfqpoint{1.642695in}{2.688480in}}%
\pgfpathlineto{\pgfqpoint{1.760228in}{2.686049in}}%
\pgfpathlineto{\pgfqpoint{1.877761in}{2.694557in}}%
\pgfpathlineto{\pgfqpoint{1.995295in}{2.687264in}}%
\pgfpathlineto{\pgfqpoint{2.112828in}{2.689695in}}%
\pgfpathlineto{\pgfqpoint{2.230361in}{2.596110in}}%
\pgfpathlineto{\pgfqpoint{2.347895in}{2.704280in}}%
\pgfpathlineto{\pgfqpoint{2.465428in}{2.754111in}}%
\pgfpathlineto{\pgfqpoint{2.582961in}{2.715218in}}%
\pgfpathlineto{\pgfqpoint{2.700495in}{2.746819in}}%
\pgfpathlineto{\pgfqpoint{2.818028in}{2.700634in}}%
\pgfpathlineto{\pgfqpoint{2.935561in}{2.724942in}}%
\pgfpathlineto{\pgfqpoint{3.053094in}{2.752895in}}%
\pgfpathlineto{\pgfqpoint{3.170628in}{2.670249in}}%
\pgfpathlineto{\pgfqpoint{3.288161in}{2.771126in}}%
\pgfpathlineto{\pgfqpoint{3.405694in}{2.654449in}}%
\pgfpathlineto{\pgfqpoint{3.523228in}{2.686049in}}%
\pgfpathlineto{\pgfqpoint{3.640761in}{2.738311in}}%
\pgfpathlineto{\pgfqpoint{3.758294in}{2.768696in}}%
\pgfpathlineto{\pgfqpoint{3.875827in}{2.741957in}}%
\pgfpathlineto{\pgfqpoint{3.993361in}{2.686049in}}%
\pgfpathlineto{\pgfqpoint{4.110894in}{2.729803in}}%
\pgfpathlineto{\pgfqpoint{4.228427in}{2.743172in}}%
\pgfpathlineto{\pgfqpoint{4.345961in}{2.729803in}}%
\pgfpathlineto{\pgfqpoint{4.463494in}{2.785711in}}%
\pgfpathlineto{\pgfqpoint{4.581027in}{2.796650in}}%
\pgfpathlineto{\pgfqpoint{4.698561in}{2.756542in}}%
\pgfpathlineto{\pgfqpoint{4.816094in}{2.775988in}}%
\pgfpathlineto{\pgfqpoint{4.933627in}{2.756542in}}%
\pgfpathlineto{\pgfqpoint{5.051160in}{2.782065in}}%
\pgfpathlineto{\pgfqpoint{5.168694in}{2.757757in}}%
\pgfpathlineto{\pgfqpoint{5.286227in}{2.789357in}}%
\pgfpathlineto{\pgfqpoint{5.403760in}{2.756542in}}%
\pgfpathlineto{\pgfqpoint{5.521294in}{2.796650in}}%
\pgfpathlineto{\pgfqpoint{5.638827in}{2.766265in}}%
\pgfpathlineto{\pgfqpoint{5.756360in}{2.778419in}}%
\pgfpathlineto{\pgfqpoint{5.873893in}{2.790573in}}%
\pgfpathlineto{\pgfqpoint{5.991427in}{2.801511in}}%
\pgfusepath{stroke}%
\end{pgfscope}%
\begin{pgfscope}%
\pgfpathrectangle{\pgfqpoint{0.561389in}{0.561666in}}{\pgfqpoint{5.688611in}{4.088334in}}%
\pgfusepath{clip}%
\pgfsetrectcap%
\pgfsetroundjoin%
\pgfsetlinewidth{1.505625pt}%
\definecolor{currentstroke}{rgb}{1.000000,0.498039,0.054902}%
\pgfsetstrokecolor{currentstroke}%
\pgfsetdash{}{0pt}%
\pgfpathmoveto{\pgfqpoint{0.819962in}{0.747499in}}%
\pgfpathlineto{\pgfqpoint{0.937495in}{1.598273in}}%
\pgfpathlineto{\pgfqpoint{1.055028in}{2.479433in}}%
\pgfpathlineto{\pgfqpoint{1.172562in}{2.564510in}}%
\pgfpathlineto{\pgfqpoint{1.290095in}{2.552356in}}%
\pgfpathlineto{\pgfqpoint{1.407628in}{2.705495in}}%
\pgfpathlineto{\pgfqpoint{1.525162in}{2.794219in}}%
\pgfpathlineto{\pgfqpoint{1.642695in}{2.672680in}}%
\pgfpathlineto{\pgfqpoint{1.760228in}{2.687264in}}%
\pgfpathlineto{\pgfqpoint{1.877761in}{2.800296in}}%
\pgfpathlineto{\pgfqpoint{1.995295in}{2.722511in}}%
\pgfpathlineto{\pgfqpoint{2.112828in}{2.899958in}}%
\pgfpathlineto{\pgfqpoint{2.230361in}{2.857419in}}%
\pgfpathlineto{\pgfqpoint{2.347895in}{2.803942in}}%
\pgfpathlineto{\pgfqpoint{2.465428in}{2.724942in}}%
\pgfpathlineto{\pgfqpoint{2.582961in}{2.889019in}}%
\pgfpathlineto{\pgfqpoint{2.700495in}{2.763834in}}%
\pgfpathlineto{\pgfqpoint{2.818028in}{2.765050in}}%
\pgfpathlineto{\pgfqpoint{2.935561in}{2.748034in}}%
\pgfpathlineto{\pgfqpoint{3.053094in}{2.808804in}}%
\pgfpathlineto{\pgfqpoint{3.170628in}{2.756542in}}%
\pgfpathlineto{\pgfqpoint{3.288161in}{2.889019in}}%
\pgfpathlineto{\pgfqpoint{3.405694in}{2.814880in}}%
\pgfpathlineto{\pgfqpoint{3.523228in}{2.765050in}}%
\pgfpathlineto{\pgfqpoint{3.640761in}{2.880512in}}%
\pgfpathlineto{\pgfqpoint{3.758294in}{2.876865in}}%
\pgfpathlineto{\pgfqpoint{3.875827in}{2.867142in}}%
\pgfpathlineto{\pgfqpoint{3.993361in}{2.813665in}}%
\pgfpathlineto{\pgfqpoint{4.110894in}{2.822173in}}%
\pgfpathlineto{\pgfqpoint{4.228427in}{2.901173in}}%
\pgfpathlineto{\pgfqpoint{4.345961in}{2.895096in}}%
\pgfpathlineto{\pgfqpoint{4.463494in}{2.862281in}}%
\pgfpathlineto{\pgfqpoint{4.581027in}{2.752895in}}%
\pgfpathlineto{\pgfqpoint{4.698561in}{2.722511in}}%
\pgfpathlineto{\pgfqpoint{4.816094in}{2.848911in}}%
\pgfpathlineto{\pgfqpoint{4.933627in}{2.803942in}}%
\pgfpathlineto{\pgfqpoint{5.051160in}{2.813665in}}%
\pgfpathlineto{\pgfqpoint{5.168694in}{2.878081in}}%
\pgfpathlineto{\pgfqpoint{5.286227in}{2.872004in}}%
\pgfpathlineto{\pgfqpoint{5.403760in}{2.835542in}}%
\pgfpathlineto{\pgfqpoint{5.521294in}{2.801511in}}%
\pgfpathlineto{\pgfqpoint{5.638827in}{2.846480in}}%
\pgfpathlineto{\pgfqpoint{5.756360in}{2.783280in}}%
\pgfpathlineto{\pgfqpoint{5.873893in}{2.763834in}}%
\pgfpathlineto{\pgfqpoint{5.991427in}{2.913327in}}%
\pgfusepath{stroke}%
\end{pgfscope}%
\begin{pgfscope}%
\pgfpathrectangle{\pgfqpoint{0.561389in}{0.561666in}}{\pgfqpoint{5.688611in}{4.088334in}}%
\pgfusepath{clip}%
\pgfsetrectcap%
\pgfsetroundjoin%
\pgfsetlinewidth{1.505625pt}%
\definecolor{currentstroke}{rgb}{0.172549,0.627451,0.172549}%
\pgfsetstrokecolor{currentstroke}%
\pgfsetdash{}{0pt}%
\pgfpathmoveto{\pgfqpoint{0.819962in}{0.747499in}}%
\pgfpathlineto{\pgfqpoint{0.937495in}{1.187471in}}%
\pgfpathlineto{\pgfqpoint{1.055028in}{2.361539in}}%
\pgfpathlineto{\pgfqpoint{1.172562in}{2.694557in}}%
\pgfpathlineto{\pgfqpoint{1.290095in}{2.885373in}}%
\pgfpathlineto{\pgfqpoint{1.407628in}{2.910896in}}%
\pgfpathlineto{\pgfqpoint{1.525162in}{2.929127in}}%
\pgfpathlineto{\pgfqpoint{1.642695in}{3.039728in}}%
\pgfpathlineto{\pgfqpoint{1.760228in}{3.056743in}}%
\pgfpathlineto{\pgfqpoint{1.877761in}{2.845265in}}%
\pgfpathlineto{\pgfqpoint{1.995295in}{3.234191in}}%
\pgfpathlineto{\pgfqpoint{2.112828in}{3.271868in}}%
\pgfpathlineto{\pgfqpoint{2.230361in}{3.183144in}}%
\pgfpathlineto{\pgfqpoint{2.347895in}{3.195298in}}%
\pgfpathlineto{\pgfqpoint{2.465428in}{3.153975in}}%
\pgfpathlineto{\pgfqpoint{2.582961in}{3.155190in}}%
\pgfpathlineto{\pgfqpoint{2.700495in}{3.206236in}}%
\pgfpathlineto{\pgfqpoint{2.818028in}{3.055528in}}%
\pgfpathlineto{\pgfqpoint{2.935561in}{3.198944in}}%
\pgfpathlineto{\pgfqpoint{3.053094in}{3.196513in}}%
\pgfpathlineto{\pgfqpoint{3.170628in}{3.296176in}}%
\pgfpathlineto{\pgfqpoint{3.288161in}{3.352084in}}%
\pgfpathlineto{\pgfqpoint{3.405694in}{3.297391in}}%
\pgfpathlineto{\pgfqpoint{3.523228in}{3.277945in}}%
\pgfpathlineto{\pgfqpoint{3.640761in}{3.224467in}}%
\pgfpathlineto{\pgfqpoint{3.758294in}{3.124805in}}%
\pgfpathlineto{\pgfqpoint{3.875827in}{3.285237in}}%
\pgfpathlineto{\pgfqpoint{3.993361in}{3.195298in}}%
\pgfpathlineto{\pgfqpoint{4.110894in}{3.237837in}}%
\pgfpathlineto{\pgfqpoint{4.228427in}{3.308330in}}%
\pgfpathlineto{\pgfqpoint{4.345961in}{3.331422in}}%
\pgfpathlineto{\pgfqpoint{4.463494in}{3.326560in}}%
\pgfpathlineto{\pgfqpoint{4.581027in}{3.297391in}}%
\pgfpathlineto{\pgfqpoint{4.698561in}{3.326560in}}%
\pgfpathlineto{\pgfqpoint{4.816094in}{3.366668in}}%
\pgfpathlineto{\pgfqpoint{4.933627in}{3.284022in}}%
\pgfpathlineto{\pgfqpoint{5.051160in}{3.325345in}}%
\pgfpathlineto{\pgfqpoint{5.168694in}{3.352084in}}%
\pgfpathlineto{\pgfqpoint{5.286227in}{3.265791in}}%
\pgfpathlineto{\pgfqpoint{5.403760in}{3.367884in}}%
\pgfpathlineto{\pgfqpoint{5.521294in}{3.319268in}}%
\pgfpathlineto{\pgfqpoint{5.638827in}{3.207452in}}%
\pgfpathlineto{\pgfqpoint{5.756360in}{3.189221in}}%
\pgfpathlineto{\pgfqpoint{5.873893in}{3.294960in}}%
\pgfpathlineto{\pgfqpoint{5.991427in}{3.377607in}}%
\pgfusepath{stroke}%
\end{pgfscope}%
\begin{pgfscope}%
\pgfpathrectangle{\pgfqpoint{0.561389in}{0.561666in}}{\pgfqpoint{5.688611in}{4.088334in}}%
\pgfusepath{clip}%
\pgfsetrectcap%
\pgfsetroundjoin%
\pgfsetlinewidth{1.505625pt}%
\definecolor{currentstroke}{rgb}{0.839216,0.152941,0.156863}%
\pgfsetstrokecolor{currentstroke}%
\pgfsetdash{}{0pt}%
\pgfpathmoveto{\pgfqpoint{0.819962in}{0.749930in}}%
\pgfpathlineto{\pgfqpoint{0.937495in}{0.924947in}}%
\pgfpathlineto{\pgfqpoint{1.055028in}{2.613125in}}%
\pgfpathlineto{\pgfqpoint{1.172562in}{3.382468in}}%
\pgfpathlineto{\pgfqpoint{1.290095in}{3.963426in}}%
\pgfpathlineto{\pgfqpoint{1.407628in}{3.899010in}}%
\pgfpathlineto{\pgfqpoint{1.525162in}{4.143304in}}%
\pgfpathlineto{\pgfqpoint{1.642695in}{4.221088in}}%
\pgfpathlineto{\pgfqpoint{1.760228in}{4.269704in}}%
\pgfpathlineto{\pgfqpoint{1.877761in}{4.283073in}}%
\pgfpathlineto{\pgfqpoint{1.995295in}{4.307381in}}%
\pgfpathlineto{\pgfqpoint{2.112828in}{4.215012in}}%
\pgfpathlineto{\pgfqpoint{2.230361in}{4.292797in}}%
\pgfpathlineto{\pgfqpoint{2.347895in}{4.171257in}}%
\pgfpathlineto{\pgfqpoint{2.465428in}{4.157888in}}%
\pgfpathlineto{\pgfqpoint{2.582961in}{4.365720in}}%
\pgfpathlineto{\pgfqpoint{2.700495in}{4.228381in}}%
\pgfpathlineto{\pgfqpoint{2.818028in}{4.326827in}}%
\pgfpathlineto{\pgfqpoint{2.935561in}{4.236889in}}%
\pgfpathlineto{\pgfqpoint{3.053094in}{4.291581in}}%
\pgfpathlineto{\pgfqpoint{3.170628in}{4.324397in}}%
\pgfpathlineto{\pgfqpoint{3.288161in}{4.334120in}}%
\pgfpathlineto{\pgfqpoint{3.405694in}{4.313458in}}%
\pgfpathlineto{\pgfqpoint{3.523228in}{3.988949in}}%
\pgfpathlineto{\pgfqpoint{3.640761in}{4.357212in}}%
\pgfpathlineto{\pgfqpoint{3.758294in}{4.241750in}}%
\pgfpathlineto{\pgfqpoint{3.875827in}{4.387597in}}%
\pgfpathlineto{\pgfqpoint{3.993361in}{4.321966in}}%
\pgfpathlineto{\pgfqpoint{4.110894in}{4.464167in}}%
\pgfpathlineto{\pgfqpoint{4.228427in}{4.341412in}}%
\pgfpathlineto{\pgfqpoint{4.345961in}{4.362074in}}%
\pgfpathlineto{\pgfqpoint{4.463494in}{4.351135in}}%
\pgfpathlineto{\pgfqpoint{4.581027in}{4.273350in}}%
\pgfpathlineto{\pgfqpoint{4.698561in}{4.432567in}}%
\pgfpathlineto{\pgfqpoint{4.816094in}{4.416767in}}%
\pgfpathlineto{\pgfqpoint{4.933627in}{4.380305in}}%
\pgfpathlineto{\pgfqpoint{5.051160in}{4.346274in}}%
\pgfpathlineto{\pgfqpoint{5.168694in}{4.366936in}}%
\pgfpathlineto{\pgfqpoint{5.286227in}{4.459305in}}%
\pgfpathlineto{\pgfqpoint{5.403760in}{4.402182in}}%
\pgfpathlineto{\pgfqpoint{5.521294in}{4.464167in}}%
\pgfpathlineto{\pgfqpoint{5.638827in}{4.460521in}}%
\pgfpathlineto{\pgfqpoint{5.756360in}{4.419197in}}%
\pgfpathlineto{\pgfqpoint{5.873893in}{4.439859in}}%
\pgfpathlineto{\pgfqpoint{5.991427in}{4.434997in}}%
\pgfusepath{stroke}%
\end{pgfscope}%
\begin{pgfscope}%
\pgfpathrectangle{\pgfqpoint{0.561389in}{0.561666in}}{\pgfqpoint{5.688611in}{4.088334in}}%
\pgfusepath{clip}%
\pgfsetrectcap%
\pgfsetroundjoin%
\pgfsetlinewidth{1.505625pt}%
\definecolor{currentstroke}{rgb}{0.580392,0.403922,0.741176}%
\pgfsetstrokecolor{currentstroke}%
\pgfsetdash{}{0pt}%
\pgfpathmoveto{\pgfqpoint{0.819962in}{0.760869in}}%
\pgfpathlineto{\pgfqpoint{0.937495in}{1.869306in}}%
\pgfpathlineto{\pgfqpoint{1.055028in}{3.325345in}}%
\pgfpathlineto{\pgfqpoint{1.172562in}{3.722778in}}%
\pgfpathlineto{\pgfqpoint{1.290095in}{3.957348in}}%
\pgfpathlineto{\pgfqpoint{1.407628in}{3.942764in}}%
\pgfpathlineto{\pgfqpoint{1.525162in}{4.024195in}}%
\pgfpathlineto{\pgfqpoint{1.642695in}{4.115349in}}%
\pgfpathlineto{\pgfqpoint{1.760228in}{4.145734in}}%
\pgfpathlineto{\pgfqpoint{1.877761in}{3.838240in}}%
\pgfpathlineto{\pgfqpoint{1.995295in}{4.278212in}}%
\pgfpathlineto{\pgfqpoint{2.112828in}{4.154242in}}%
\pgfpathlineto{\pgfqpoint{2.230361in}{4.144519in}}%
\pgfpathlineto{\pgfqpoint{2.347895in}{4.229596in}}%
\pgfpathlineto{\pgfqpoint{2.465428in}{4.205289in}}%
\pgfpathlineto{\pgfqpoint{2.582961in}{4.201642in}}%
\pgfpathlineto{\pgfqpoint{2.700495in}{4.208935in}}%
\pgfpathlineto{\pgfqpoint{2.818028in}{4.094688in}}%
\pgfpathlineto{\pgfqpoint{2.935561in}{4.229596in}}%
\pgfpathlineto{\pgfqpoint{3.053094in}{3.988949in}}%
\pgfpathlineto{\pgfqpoint{3.170628in}{4.353566in}}%
\pgfpathlineto{\pgfqpoint{3.288161in}{4.279427in}}%
\pgfpathlineto{\pgfqpoint{3.405694in}{4.307381in}}%
\pgfpathlineto{\pgfqpoint{3.523228in}{4.275781in}}%
\pgfpathlineto{\pgfqpoint{3.640761in}{4.200427in}}%
\pgfpathlineto{\pgfqpoint{3.758294in}{4.289150in}}%
\pgfpathlineto{\pgfqpoint{3.875827in}{4.309812in}}%
\pgfpathlineto{\pgfqpoint{3.993361in}{4.320751in}}%
\pgfpathlineto{\pgfqpoint{4.110894in}{4.269704in}}%
\pgfpathlineto{\pgfqpoint{4.228427in}{4.250258in}}%
\pgfpathlineto{\pgfqpoint{4.345961in}{4.358428in}}%
\pgfpathlineto{\pgfqpoint{4.463494in}{4.352351in}}%
\pgfpathlineto{\pgfqpoint{4.581027in}{4.388813in}}%
\pgfpathlineto{\pgfqpoint{4.698561in}{4.348704in}}%
\pgfpathlineto{\pgfqpoint{4.816094in}{4.332905in}}%
\pgfpathlineto{\pgfqpoint{4.933627in}{4.373012in}}%
\pgfpathlineto{\pgfqpoint{5.051160in}{4.370581in}}%
\pgfpathlineto{\pgfqpoint{5.168694in}{4.341412in}}%
\pgfpathlineto{\pgfqpoint{5.286227in}{4.369366in}}%
\pgfpathlineto{\pgfqpoint{5.403760in}{4.353566in}}%
\pgfpathlineto{\pgfqpoint{5.521294in}{4.351135in}}%
\pgfpathlineto{\pgfqpoint{5.638827in}{4.368151in}}%
\pgfpathlineto{\pgfqpoint{5.756360in}{4.352351in}}%
\pgfpathlineto{\pgfqpoint{5.873893in}{4.362074in}}%
\pgfpathlineto{\pgfqpoint{5.991427in}{4.328043in}}%
\pgfusepath{stroke}%
\end{pgfscope}%
\begin{pgfscope}%
\pgfsetrectcap%
\pgfsetmiterjoin%
\pgfsetlinewidth{0.803000pt}%
\definecolor{currentstroke}{rgb}{0.000000,0.000000,0.000000}%
\pgfsetstrokecolor{currentstroke}%
\pgfsetdash{}{0pt}%
\pgfpathmoveto{\pgfqpoint{0.561389in}{0.561666in}}%
\pgfpathlineto{\pgfqpoint{0.561389in}{4.650000in}}%
\pgfusepath{stroke}%
\end{pgfscope}%
\begin{pgfscope}%
\pgfsetrectcap%
\pgfsetmiterjoin%
\pgfsetlinewidth{0.803000pt}%
\definecolor{currentstroke}{rgb}{0.000000,0.000000,0.000000}%
\pgfsetstrokecolor{currentstroke}%
\pgfsetdash{}{0pt}%
\pgfpathmoveto{\pgfqpoint{6.250000in}{0.561666in}}%
\pgfpathlineto{\pgfqpoint{6.250000in}{4.650000in}}%
\pgfusepath{stroke}%
\end{pgfscope}%
\begin{pgfscope}%
\pgfsetrectcap%
\pgfsetmiterjoin%
\pgfsetlinewidth{0.803000pt}%
\definecolor{currentstroke}{rgb}{0.000000,0.000000,0.000000}%
\pgfsetstrokecolor{currentstroke}%
\pgfsetdash{}{0pt}%
\pgfpathmoveto{\pgfqpoint{0.561389in}{0.561666in}}%
\pgfpathlineto{\pgfqpoint{6.250000in}{0.561666in}}%
\pgfusepath{stroke}%
\end{pgfscope}%
\begin{pgfscope}%
\pgfsetrectcap%
\pgfsetmiterjoin%
\pgfsetlinewidth{0.803000pt}%
\definecolor{currentstroke}{rgb}{0.000000,0.000000,0.000000}%
\pgfsetstrokecolor{currentstroke}%
\pgfsetdash{}{0pt}%
\pgfpathmoveto{\pgfqpoint{0.561389in}{4.650000in}}%
\pgfpathlineto{\pgfqpoint{6.250000in}{4.650000in}}%
\pgfusepath{stroke}%
\end{pgfscope}%
\begin{pgfscope}%
\pgfsetbuttcap%
\pgfsetmiterjoin%
\definecolor{currentfill}{rgb}{1.000000,1.000000,1.000000}%
\pgfsetfillcolor{currentfill}%
\pgfsetfillopacity{0.800000}%
\pgfsetlinewidth{1.003750pt}%
\definecolor{currentstroke}{rgb}{0.800000,0.800000,0.800000}%
\pgfsetstrokecolor{currentstroke}%
\pgfsetstrokeopacity{0.800000}%
\pgfsetdash{}{0pt}%
\pgfpathmoveto{\pgfqpoint{4.420139in}{0.631111in}}%
\pgfpathlineto{\pgfqpoint{6.152778in}{0.631111in}}%
\pgfpathquadraticcurveto{\pgfqpoint{6.180556in}{0.631111in}}{\pgfqpoint{6.180556in}{0.658888in}}%
\pgfpathlineto{\pgfqpoint{6.180556in}{1.651742in}}%
\pgfpathquadraticcurveto{\pgfqpoint{6.180556in}{1.679520in}}{\pgfqpoint{6.152778in}{1.679520in}}%
\pgfpathlineto{\pgfqpoint{4.420139in}{1.679520in}}%
\pgfpathquadraticcurveto{\pgfqpoint{4.392361in}{1.679520in}}{\pgfqpoint{4.392361in}{1.651742in}}%
\pgfpathlineto{\pgfqpoint{4.392361in}{0.658888in}}%
\pgfpathquadraticcurveto{\pgfqpoint{4.392361in}{0.631111in}}{\pgfqpoint{4.420139in}{0.631111in}}%
\pgfpathclose%
\pgfusepath{stroke,fill}%
\end{pgfscope}%
\begin{pgfscope}%
\pgfsetrectcap%
\pgfsetroundjoin%
\pgfsetlinewidth{1.505625pt}%
\definecolor{currentstroke}{rgb}{0.121569,0.466667,0.705882}%
\pgfsetstrokecolor{currentstroke}%
\pgfsetdash{}{0pt}%
\pgfpathmoveto{\pgfqpoint{4.447916in}{1.575353in}}%
\pgfpathlineto{\pgfqpoint{4.725694in}{1.575353in}}%
\pgfusepath{stroke}%
\end{pgfscope}%
\begin{pgfscope}%
\definecolor{textcolor}{rgb}{0.000000,0.000000,0.000000}%
\pgfsetstrokecolor{textcolor}%
\pgfsetfillcolor{textcolor}%
\pgftext[x=4.836805in,y=1.526742in,left,base]{\color{textcolor}\rmfamily\fontsize{10.000000}{12.000000}\selectfont Trained on v2.1}%
\end{pgfscope}%
\begin{pgfscope}%
\pgfsetrectcap%
\pgfsetroundjoin%
\pgfsetlinewidth{1.505625pt}%
\definecolor{currentstroke}{rgb}{1.000000,0.498039,0.054902}%
\pgfsetstrokecolor{currentstroke}%
\pgfsetdash{}{0pt}%
\pgfpathmoveto{\pgfqpoint{4.447916in}{1.376187in}}%
\pgfpathlineto{\pgfqpoint{4.725694in}{1.376187in}}%
\pgfusepath{stroke}%
\end{pgfscope}%
\begin{pgfscope}%
\definecolor{textcolor}{rgb}{0.000000,0.000000,0.000000}%
\pgfsetstrokecolor{textcolor}%
\pgfsetfillcolor{textcolor}%
\pgftext[x=4.836805in,y=1.327576in,left,base]{\color{textcolor}\rmfamily\fontsize{10.000000}{12.000000}\selectfont Trained on v3a}%
\end{pgfscope}%
\begin{pgfscope}%
\pgfsetrectcap%
\pgfsetroundjoin%
\pgfsetlinewidth{1.505625pt}%
\definecolor{currentstroke}{rgb}{0.172549,0.627451,0.172549}%
\pgfsetstrokecolor{currentstroke}%
\pgfsetdash{}{0pt}%
\pgfpathmoveto{\pgfqpoint{4.447916in}{1.177021in}}%
\pgfpathlineto{\pgfqpoint{4.725694in}{1.177021in}}%
\pgfusepath{stroke}%
\end{pgfscope}%
\begin{pgfscope}%
\definecolor{textcolor}{rgb}{0.000000,0.000000,0.000000}%
\pgfsetstrokecolor{textcolor}%
\pgfsetfillcolor{textcolor}%
\pgftext[x=4.836805in,y=1.128409in,left,base]{\color{textcolor}\rmfamily\fontsize{10.000000}{12.000000}\selectfont Trained on v3b}%
\end{pgfscope}%
\begin{pgfscope}%
\pgfsetrectcap%
\pgfsetroundjoin%
\pgfsetlinewidth{1.505625pt}%
\definecolor{currentstroke}{rgb}{0.839216,0.152941,0.156863}%
\pgfsetstrokecolor{currentstroke}%
\pgfsetdash{}{0pt}%
\pgfpathmoveto{\pgfqpoint{4.447916in}{0.977854in}}%
\pgfpathlineto{\pgfqpoint{4.725694in}{0.977854in}}%
\pgfusepath{stroke}%
\end{pgfscope}%
\begin{pgfscope}%
\definecolor{textcolor}{rgb}{0.000000,0.000000,0.000000}%
\pgfsetstrokecolor{textcolor}%
\pgfsetfillcolor{textcolor}%
\pgftext[x=4.836805in,y=0.929243in,left,base]{\color{textcolor}\rmfamily\fontsize{10.000000}{12.000000}\selectfont Trained on v3}%
\end{pgfscope}%
\begin{pgfscope}%
\pgfsetrectcap%
\pgfsetroundjoin%
\pgfsetlinewidth{1.505625pt}%
\definecolor{currentstroke}{rgb}{0.580392,0.403922,0.741176}%
\pgfsetstrokecolor{currentstroke}%
\pgfsetdash{}{0pt}%
\pgfpathmoveto{\pgfqpoint{4.447916in}{0.778688in}}%
\pgfpathlineto{\pgfqpoint{4.725694in}{0.778688in}}%
\pgfusepath{stroke}%
\end{pgfscope}%
\begin{pgfscope}%
\definecolor{textcolor}{rgb}{0.000000,0.000000,0.000000}%
\pgfsetstrokecolor{textcolor}%
\pgfsetfillcolor{textcolor}%
\pgftext[x=4.836805in,y=0.730077in,left,base]{\color{textcolor}\rmfamily\fontsize{10.000000}{12.000000}\selectfont Trained on v3\(\displaystyle _{\textrm{no-noise}}\)}%
\end{pgfscope}%
\end{pgfpicture}%
\makeatother%
\endgroup%
}
  \caption[Pre-training learning curves]{Learning curves on the development set
    of \nlmapsthree{}}
  \label{fig:pre-trained-learning-curve}
\end{figure}

\section{Annotation for New Dataset}
\label{sec:annotation}

Training and evaluating useful \nlmaps{} models without having a dataset of
actual user queries is problematic. For this reason, an annotation experiment is
conducted, during which several people use our web interface presented in
Chapter~\ref{ch:web-interface} to issue new NL queries. They are tasked to
confirm the model’s predicted MRL query if they deem it correct or to correct it
if they consider it false. Before beginning the task, the annotators complete a
tutorial,\footnote{\url{https://nlmaps.gorgor.de/tutorial}.} which explains
the basics of OSM, the capabilities of the MRL language and the usage of the web
interface.

The call for participation in the annotation study was distributed through
various OSM-typical channels including the \emph{talk} mailing
list\footcite{osm-talk} and the OSM subreddit\footcite{osm-subreddit}, which
yielded the highest response. Albeit slightly eurocentric, the participants stem
from a fairly diverse set of countries and feature. In contrast, all of the
acquired annotators report their gender as male, which is reflective of the
overall OSM demographic since only between \SI{2}{\%} and \SI{5}{\%} of OSM
contributors are estimated to be female
\parencites{budhathoki-2010}{stark-2010}{lechner-2011}{klettner-2013}{das-2019}.
This is problematic as it has the potential to lead to a bias in the dataset. In
future annotation projects this effect should be mitigated by targeted
advertising for projects in female mapping communities such as GeoChicas.\footcite{geochicas}

\begin{table}[h]
  \centering
  \resizebox{\textwidth}{!}{%
    \begin{tabular}[t]{lllll} % 63
      \toprule
      \#Annotations & Nationality & Native Language & OSM Experience & Gender\\
      \midrule
      \num{442} & Poland & Polish & Extensive & Male\\ % Mateusz
      \num{414} & Germany & German & Extensive & Male\\ % Martin
      \num{405} & Turkey & Turkish & Medium & Male\\ % Ali Y
      \num{404} & UK & English & Medium & Male\\ % Jake
      \num{404} & Hungary & Hungarian & Extensive & Male\\ % Benjámin Z
      \num{400} & India & Hindi & Little & Male\\ % Aksha
      \num{400} & Turkey & Turkish & Little & Male\\ % Ali A
      \num{393} & Germany & German & Little & Male\\ % David
      \num{361} & Germany & German & Extensive & Male\\ % Simon
      \num{253} & Brazil & Portuguese & Little & Male\\ % Thales
      \num{253} & Nepal & Nepalese & Extensive & Male\\ % Rabin
      \num{19} & Philippines & English, Filipino, Cebuano & Extensive & Male\\ % Erwin
      \num{43} & \multicolumn{4}{l}{\emph{Various untracked people without login}}\\
      \bottomrule
    \end{tabular}}
  \caption[Annotator profiles]{Information about annotators. Everything but the
    number of annotations is self-reported by the annotators.}
  \label{tab:annotator-profiles}
\end{table}

The annotation guidelines explain to the annotators that their queries should be
linguistically diverse, cover a large part of OSM tags and cover more useful
tags in more detail. To ensure that the annotators sufficiently cover the most
important tags, we choose a set of tags (e.g. \osmtag{amenity=cafe} or
\osmtag{wheelchair=yes}) and keys (e.g. \osmtag{shop} or \osmtag{leisure}) for
which each annotator must issue a minimum number of queries. See the full
guidelines in Appendix~\ref{ch:annotation-guidelines} for details. The
annotators can monitor their annotation progress via an overview page in the web
interface, which is shown in Figure~\ref{fig:annotation-progress}.

\begin{figure}[h]
  \centering
  \includegraphics[width=\textwidth]{fig/annotation_progress.png}
  \caption[Annotation progress overview]{Overview of an annotator’s annotation
    progress}
  \label{fig:annotation-progress}
\end{figure}

\section{Online Learning Simulation}
\label{sec:online-simulation}

%%% Local Variables:
%%% coding: utf-8
%%% mode: latex
%%% TeX-engine: xetex
%%% TeX-parse-self: t
%%% TeX-command-extra-options: "-shell-escape"
%%% TeX-master: "../thesis"
%%% End: