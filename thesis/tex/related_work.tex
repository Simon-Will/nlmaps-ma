\chapter{Related Work}
\label{ch:related-work}

\section{Natural Language Interfaces}

% TODO: Intro to Natural Language Interfaces
% Goals, first interfaces, intent classification, etc.
% Purposes (see wikipedia), my purpose: querying

% Q&A Systems, closed vs open domain

% SQL Parsing

\subsection{OpenStreetMap Query Systems}
% Nominatim, Overpass, Sophox, Overpass Turbo Wizard

\subsubsection{OpenStreetMap and its Ecosystem}
\label{sec:osm}

In a crowd-sourcing approach similar to Wikipedia’s, the
OpenStreetMap\footcite{openstreetmap} (OSM) project aims to create a map of the
world by letting users contribute missing data, ranging from low-granular
objects like forests or streets to high-granular objects like benches or
information boards and even including non-geographical information like opening
hours of stores or the types of cuisine available in restaurants. The
OpenStreetMap Foundation makes the map data available under the Open Data
Commons Open Database License\footcite{odbl} effectively allowing the usage of
the data for any project but requiring that any extensions of the data are
shared under the same license again.

OpenStreetMap data is made up of three different elements: Nodes, ways (ordered
lists of nodes) and relations (groups of elements). These elements’ meaning is
derived from the tags that are added to them. For instance an Italian restaurant
that has vegan options and is wheelchair-accessible may be tagged with the
following tags:

\begin{itemize}
\item \osmtag{amenity=restaurant}
\item \osmtag{cuisine=italian}
\item \osmtag{diet:vegan=yes}
\item \osmtag{diet:vegetarian=yes}
\item \osmtag{wheelchair=yes}
\item \osmtag{opening_hours=Mo-Sa 11:30-22:00}
\item \osmtag{website=https://restaurant.example.com/}
\end{itemize}

The OpenStreetMap database can be queried in a number of ways, the most
prevalent one of which is via Geocoders such as Nominatim \footcite{nominatim}.
They allow the database being queried by the name or address of an object in
\emph{forward geocoding} or by its geographic coordinates in \emph{reverse
  geocoding}.

For querying by more than name or address, there are two specialized systems:
Sophox\footcite[The official website at \url{https://sophox.org} is offline as
of December 2020]{sophox} and the Overpass API \footcite{overpass-api}, which
can be used most conveniently via the Overpass Turbo \footcite{overpass-turbo}
interface. The Overpass API allows queries using an XML-like language or – more
prominently – its custom Overpass Query Language (Overpass QL). The question
\nl{Which Italian restaurants in Berlin are wheelchair-accessible?} could be
expressed with the following Overpass Query

\begin{lstlisting}[style=MyOverpassQL,title={Overpass QL for wheelchair-accessible restaurants in Heidelberg}]
(area[name=Heidelberg];) -> .a;
nwr[amenity=restaurant][wheelchair=yes](area.a);
out geom;
\end{lstlisting}

\subsubsection{NLMaps}
\label{sec:nlmaps}

The open license as well as the diverse and rich information available in the
data make OSM a promising candidate for the foundation of an information
retrieval system about questions pertaining to geo-related information. The
first step in this direction was made by \textcite{haas-2016}, when they
released the first version of the \nlmaps{} dataset a \textquote[][.]{corpus
  consisting of 2,380 questions about geographical facts that can be answered
  with the [OSM] database}\footcite{nlmaps} Each question is provided as
a natural language (NL) query in English and in German and as its rendering in a
custom machine-readable language (MRL, see Section~\ref{sec:mrl}) query. The
dataset can be used to develop a parser for parsing an NL query into its
corresponding MRL query, which can then be used to extract the answer to the
question from the OSM database.

In two subsequent works, Lawrence\footnote{Carolin Haas changed her name to
  Carolin Lawrence in 2016.} and Riezler
\parencites*{lawrence-2016}{lawrence-2018} expanded the English
part\footnote{\nlmapstwo{} is not available in German.} of the dataset to
include more NL-MRL pairs and by extension also more word types and OSM tags
(see Section~\ref{sec:mrl}). Table~\ref{tab:nlmaps-v1-v2-stats} shows key data
about the size of the extended dataset. Table~\ref{tab:nlmapsv2-splits} shows
the size of the dataset splits. In contrast to the first version, the NL-MRL
pairs in this extended version were created with a templating approach, which
made use of a
table\footnote{\url{https://wiki.openstreetmap.org/wiki/Nominatim/Special_Phrases/EN}.
  It is not known which version of the table was used for generating the
  \nlmapstwo{} dataset.} mapping natural language expressions (e.g.
“restaurant”) to OSM tags (e.g. \osmtag{amenity=restaurant}).

In addition to the NL and the MRL queries, the dataset includes a linearized
(LIN) version of the MRL query. This is a formally equivalent variant of the MRL
query that avoids parentheses and commata by specifying each operator’s arity
instead. For further information on this,
\textcites(cf.)(){andreas-2013}{haas-2016}. All parsing models discussed in this
thesis parse the NL query into the LIN query, which can be converted into the
MRL query for retrieving the result from the OSM database.

All of the question and query variants are also provided in a version where the
locations and the points of interest are replaced by generic
\lstinline!_LOCATION! and \lstinline!_POI! tokens, respectively. This is
intended to simplify training a parser model which relies on an external Named
Entity Recognition (NER) component for the named entities.

\textcite{lawrence-2018} trained a GRU-based encoder-decoder model
\parencite{cho-2014} with Bahdanau attention \parencite{bahdanau-2015} on
\nlmapstwo{}, once without masking the named entities and once with masking
them. The model trained on the masked version is accompanied by an NER model.
\textcite{staniek-2020} trained a similar model for comparison with
\textcite{lawrence-2018}, once as a token-based RNN and once as a
character-based RNN, both without masking the named entities. The results are
shown in Table~\ref{tab:lawrence-staniek-results}. In essence, they show that it
is easy enough for the character-based model to copy the named entities from the
source to the target so that a separate NER model does not improve the results.

\begin{table}[ht!]
  \centering
  \begin{tabular}{lrr}
    \toprule
    Model & unmasked & masked + NER\\
    \midrule
    \textcite{lawrence-2018} (token) & \num{.804} & \num{.901}\\
    \textcite{staniek-2020} (token) & \num{.834} & ---\\
    \textcite{staniek-2020} (character) & \bfnum{.938} & ---\\
    \bottomrule
  \end{tabular}
  \caption[Previous NLMaps results]{Accuracy of models by
    \textcite{lawrence-2018} and \textcite{staniek-2020} on \nlmapstwo{}}
  \label{tab:lawrence-staniek-rsults}
\end{table}

An accuracy of \SI{93.8}{\%} makes it seem as though the task of parsing NL
queries were mostly solved. However, this is not the case, at all. The high
accuracy is the result of a number of shortcomings in the \nlmapstwo{} dataset ,
a part of which has already been noticed by \textcite{staniek-2020} and which
are investigated in more detail in Chapter~\ref{ch:nlmaps-improvement}.

\begin{table}[ht!]
  \centering
  \begin{tabular}[h]{lll}
    \toprule
    & \nlmapsone{} & \nlmapstwo{}\\
    \midrule
    question-query-pairs & \num{2380} & \num{28609}\\
    tokens & \num{25906} & \num{202088}\\
    types & \num{1002} & \num{8710}\\
    avg.\ sentence length & \num{10.88} & \num{7.06}\\
    distinct tags & \num{477} & \num{6582}\\
    \bottomrule
  \end{tabular}
  \caption[\nlmapstwo{} statistics]{Numeric information about \nlmapsone{} and
    \nlmapstwo{}. The table is reproduced from \textcite{lawrence-2018}.}
  \label{tab:nlmaps-v1-v2-stats}
\end{table}

\begin{table}[ht!]
  \centering
  \begin{tabular}[h]{ll}
    \toprule
    Set split & \nlmapstwo{}\\
    \midrule
    train & \num{16172}\\
    dev & \num{1843}\\
    test & \num{10594}\\
    \bottomrule
  \end{tabular}
  \caption[\nlmapstwo{} splits]{Split sizes in the \nlmapstwo{} dataset}
  \label{tab:nlmapsv2-splits}
\end{table}

\section{Online Learning}

%%% Local Variables:
%%% coding: utf-8
%%% mode: latex
%%% TeX-engine: xetex
%%% TeX-parse-self: t
%%% TeX-command-extra-options: "-shell-escape"
%%% TeX-master: "../thesis"
%%% End:
