\chapter{NLMaps Data Improvement}
\label{ch:nlmaps-improvement}

\section{Analysis of \nlmapstwo{}}

After training a character-based encoder-decoder model as described by
\textcite{staniek-2020} on the \nlmapstwo{} dataset, it quickly becomes
apparent that the \SI{93.8}{\%} performance on the test split is not reflected
in the model’s performance on new queries. Figure~\ref{fig:nlmaps2-reality-check}

Six separate issues with the \nlmapstwo{} dataset can be identified that lead to
a subpar performance on new queries not present in the training set or test set.

\begin{enumerate}
\item Extremely close resemblance between training set and test set
\item Inconsistencies in mapping from NL term to OSM tag
\item Inconsistencies in MRL syntax
\item Little linguistic variety on the NL side
\item Little variety with respect to location names
\item Usage of deprecated OSM tags
\end{enumerate}

\subsection{Train/Test Resemblance}

As already noticed by \textcite{staniek-2020}, the fact that \nlmapstwo{} was
created by using quite simple templates led to nearly the same NL query
occurring in the training and test set – only the named entities of the location
and the point of interest (if any) being different. E.g. the training set
contains the query \nl{where book store in Heidelberg}, while the test set
contains the two queries \nl{where book store in Edinburgh} and \nl{where book
  store in Paris}.

By removing all queries from the development and test sets that appear
identically in the training when disregarding the named entities,
\textcite{staniek-2020} reduced the size of the test set from \num{10594}
to \num{4156} queries. On this smaller test set, his model’s accuracy fell from
\SI{93.8}{\%} to \SI{83.5}{\%}.

It must be noted that even though the most glaring similarities between the
training and the test set can be removed in this way, the underlying reason for
the similarity remains: Both are generated by the same templates using the same
table for mapping NL terms to OSM tags. As a consequence, only 11 of
534 tags (already excluding \osmtag{name=*} tags) in the test set do not occur
in the training set.\footnote{And 4 of those are proper names. The 11 tags are:
  \osmtag{addr:street=Bergheimer Straße}, \osmtag{brand=Vauxhall},
  \osmtag{cuisine=german}, \osmtag{fireplace=yes},
  \osmtag{internet_access:fee=no}, \osmtag{product=whisky}, \osmtag{ref=A 4},
  \osmtag{ref=M90}, \osmtag{school:de=Grundschule},
  \osmtag{shelter_type=weather_shelter}, \osmtag{sports=tennis}}

While it is theoretically possible to split off a number of templates, terms and
OSM tags for generating a independent test set, the templating engine will still
remain the same and the templates may also be thought up by the same template
author. A robust evaluation should happen on human-written instead of
machine-generated queries.

\subsection{Inconsistencies in NL Term to Tag Mapping}

There is a collaborative table on the OSM Wiki that maps NL terms to OSM
tags,\footnote{\url{https://wiki.openstreetmap.org/wiki/Nominatim/Special_Phrases/EN}.}
whose structure is shown in a simplified way in the excerpt provided in
Table~\ref{tab:special-phrases-excerpt}. For generating an NL-MRL pair of
\nlmapstwo{}, a row of the table was selected, the NL term was put into a
template for the NL query and the OSM tag was used for building the MRL. For
terms which are mapped to only one OSM tag in the table, this approach works
fine. However, there are terms like \emph{forest} or \emph{bar} which are mapped
to two different OSM tags. This leads to the situation that in \nlmapstwo{}, an
NL query asking for a \emph{pub} may have \osmtag{amenity=pub} in the
corresponding MRL and the next query asking for a \emph{pub} may have
\osmtag{amenity=bar} instead. This is of course impossible to learn for a model.

\begin{table}[ht!]
  \centering
  \begin{tabular}{ll}
    NL Term & OSM Tag\\
    airport & \osmtag{aeroway=aerodrome}\\
    bar & \osmtag{amenity=bar}\\
    bar & \osmtag{amenity=pub}\\
    church & \osmtag{amenity=place_of_worship}\\
    forest & \osmtag{landuse=forest}\\
    forest & \osmtag{natural=wood}\\
    pub & \osmtag{amenity=bar}\\
    pub & \osmtag{amenity=pub}\\
    wood & \osmtag{landuse=forest}\\
    wood & \osmtag{natural=wood}\\
  \end{tabular}
  \caption[Special Phrases]{Simplified Excerpt from Special Phrases Table}
  \label{tab:special-phrases-excerpt}
\end{table}

The solution for this requires some insight into the OSM tags in question. In
cases like \osmtag{landuse=forest} and \osmtag{natural=wood}, the user issuing
the query most likely will not care about the
difference,\footnote{\osmtag{landuse=forest} is mostly used for areas managed
  for forestry while \osmtag{natural=wood} is used for wild forests. However,
  the situation differs across mappers.
  \url{https://wiki.openstreetmap.org/wiki/Forest}.} so they should be merged
into \mrl{or(keyval('landuse','forest'),keyval('natural','wood'))} in the MRL.
In other cases, the user may care about the difference: The bar-pub distinction
is fairly transparent and a user asking for pubs should not be referred to
bars.

\subsection{Inconsistencies in MRL Syntax}

When querying for things around a place, it’s possible to specify a name for the
reference place with the \mrl{nwr} operator as well as the area which that
reference place is located in with the \mrl{area} operator. A typical MRL is
shown in Figure~\ref{fig:around-with-both}.

\begin{figure}[ht!]
  \centering
  \begin{lstlisting}[style=MyMRL]
    query(
      around(
        center(
          area(keyval('name','Liverpool')),
          nwr(keyval('name','Mollington Avenue'))
        ),
        search(nwr(keyval('amenity','bank'))),
        maxdist(DIST_INTOWN),
        topx(1)
      ),
      qtype(latlong)
    )
  \end{lstlisting}
  \caption{The MRL for \nl{closest Bank from Mollington Avenue in Liverpool} has
    both the \mrl{nwr} and \mrl{area} operators in the \mrl{center} clause}
  \label{fig:around-with-both}
\end{figure}

When however the reference place is given without specifying an area which it is
located in, some MRLs have the reference place in the \mrl{nwr} operator while
others have it in the \mrl{area} operator. This is a meaningless difference and
impossible for the model to learn consistently. The easiest way to resolve this
is by just replacing the \mrl{area} operator with the \mrl{nwr} operator when
the \mrl{center} clause has no \mrl{nwr} operator. Examples are shown in
Figure~\ref{fig:around-with-one}.

\begin{figure}[ht!]
  \centering
  \begin{subfigure}{\textwidth}
    \begin{lstlisting}[style=MyMRL]
      query(
        around(
          center(
            area(keyval('name','Heidelberg'))
          ),
          search(nwr(keyval('place','town'))),
          maxdist(DIST_OUTTOWN)
        ),
        qtype(count)
      )
    \end{lstlisting}
    \caption{\nl{how many towns around Heidelberg}}
  \end{subfigure}
  \begin{subfigure}{\textwidth}
    \begin{lstlisting}[style=MyMRL]
      query(
        around(
          center(
            nwr(keyval('name','Nantes'))
          ),
          search(nwr(keyval('amenity','waste_basket'))),
          maxdist(DIST_INTOWN)
        ),
        qtype(count)
      )
    \end{lstlisting}
    \caption{\nl{How many Rubbish Bins near Nantes}}
  \end{subfigure}
  \caption{Inconsistent use of \mrl{nwr} and \mrl{area} operator for reference
    place in two MRLs}
  \label{fig:around-with-both}
\end{figure}


\subsection{Usage of deprecated OSM tags}

%%% Local Variables:
%%% coding: utf-8
%%% mode: latex
%%% TeX-engine: xetex
%%% TeX-parse-self: t
%%% TeX-command-extra-options: "-shell-escape"
%%% TeX-master: "../thesis"
%%% End: